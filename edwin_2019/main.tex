%!Mode:: "TeX:System

%%%%%%%%%%%%%%%%%%%%%%%%%%%%%%%%%%%%%%%%%%%%%%%%%%%%%%%%%%%%%%%%%%%%%%%%%%%%%
%                                                                           %
%          LaTeX File for Doctor (Master) Thesis of ECNU                    %
%            华东师范大学博士(硕士)论文模板 ____lizb                      %
%                                                                           %
%%%%%%%%%%%%%%%%%%%%%%%%%%%%%%%%%%%%%%%%%%%%%%%%%%%%%%%%%%%%%%%%%%%%%%%%%%%%%




\documentclass[12pt,openany,a4paper,fancyhdr,twoside]{ctexbook}

%draft 选项可以使插入的图形只显示外框,以加快预览速度。
%\documentclass[11pt,a4paper,openany,draft]{book}
\usepackage{amsmath}
\usepackage{amssymb}               % AMSLaTeX宏包 用来排出更加漂亮的公式
%\usepackage[CJKbookmarks]{hyperref}
\usepackage{url}
\usepackage[hidelinks]{hyperref}
\usepackage{shortvrb,indentfirst,ulem,makeidx}
\usepackage{fancyhdr}
\usepackage{graphicx}

\usepackage{rotating}


\usepackage{indentfirst,latexsym,colortbl,subfigure,clrscode}

%\usepackage[ruled,vlined,linesnumbered,]{algorithm2e}

\usepackage[linesnumbered,ruled,vlined,resetcount,algochapter]{algorithm2e}%[ruled,vlined]{
\usepackage{algorithmicx}
\usepackage{listings}
\usepackage{algcompatible}
\usepackage{algpseudocode}


\renewcommand*{\algorithmcfname}{算法}

\newcommand{\clearPaperPage}{\clearpage}
%$\renewcommand{\clearPaperPage}{\cleardoublepage}




\SetKwInput{KwIn}{\textbf{输入}}
\SetKwInput{KwOut}{\textbf{输出}}
\SetKwInput{KwReturn}{\textbf{return}}


\usepackage{bibspacing}

\setlength{\bibitemsep}{2\baselineskip plus .05\baselineskip minus .05\baselineskip}


\usepackage{bm}                     % 处理数学公式中的黑斜体的宏包
\usepackage{amssymb}                % AMSLaTeX宏包 用来排出更加漂亮的公式
\usepackage{mathrsfs}
\usepackage[subnum]{cases}
%\usepackage[numbers,sort&compress]{natbib}
\usepackage[super,square,comma,sort&compress]{natbib}
\usepackage{hypernat}
\usepackage{geometry}
\usepackage{times}
\usepackage{fontspec}
%\usepackage{libertine}
\usepackage{libertineotf}
\usepackage{caption}
\usepackage{titletoc}
\usepackage{mathtools}

%\usepackage{chngcntr}
\counterwithout{footnote}{chapter}



%\usepackage{cite}
\usepackage{longtable,booktabs}
\usepackage{multirow}
\usepackage{subfigure}
\usepackage[subfigure]{tocloft}

\usepackage{float}
\usepackage{balance}
\usepackage {paralist}
\usepackage{bbding}
\usepackage{pgffor}

\usepackage{threeparttable}


%\usepackage{biblatex}
\makeindex
\pagestyle{fancy}

\newcommand{\eat}[1]{}

\renewcommand{\headrulewidth}{0.4pt}
\fancyfoot[CO,CE]{\thepage}

\renewcommand{\algorithmicrequire}{\textbf{Input:}}
\renewcommand{\algorithmicensure}{\textbf{Output:}}

\newtheorem{Def}{定义}


\newcommand{\yihao}{\fontsize{26pt}{36pt}\selectfont}           % 一号, 1.4 倍行距
\newcommand{\erhao}{\fontsize{22pt}{28pt}\selectfont}          % 二号, 1.25 倍行距
\newcommand{\xiaoer}{\fontsize{18pt}{18pt}\selectfont}          % 小二, 单倍行距
\newcommand{\sanhao}{\fontsize{16pt}{24pt}\selectfont}        % 三号, 1.5 倍行距
\newcommand{\xiaosan}{\fontsize{15pt}{22pt}\selectfont}        % 小三, 1.5 倍行距
\newcommand{\sihao}{\fontsize{14pt}{21pt}\selectfont}            % 四号, 1.5 倍行距
\newcommand{\banxiaosi}{\fontsize{13pt}{19.5pt}\selectfont}    % 半小四, 1.5 倍行距
\newcommand{\xiaosi}{\fontsize{12pt}{18pt}\selectfont}            % 小四, 1.5 倍行距
\newcommand{\dawuhao}{\fontsize{11pt}{11pt}\selectfont}       % 大五号, 单倍行距
\newcommand{\wuhao}{\fontsize{10.5pt}{15.75pt}\selectfont}    % 五号, 单倍行距

\newcommand{\tablewuhao}{\fontsize{10.5pt}{12.5pt}\selectfont}    % 五号, 单倍行距

\newcommand{\equwuhao}{\fontsize{11.5pt}{15pt}\selectfont}

\renewcommand*{\bibfont}{\normalfont\normalsize\linespread{1}\selectfont}
\setlength{\bibitemsep}{\baselineskip}



%============================ 可以自定义文字块 ================================%


%交叉引用格式
\renewcommand\figureautorefname{图}
\renewcommand\tableautorefname{表}
\renewcommand\equationautorefname{式子}
\renewcommand{\algorithmautorefname}{算法}





\renewcommand{\contentsname}{\hfill\bfseries\Large 目录\hfill}
\renewcommand{\cftaftertoctitle}{\hfill}

\renewcommand{\listtablename}{\hfill\bfseries\Large 表~~~~格\hfill}
\renewcommand{\listfigurename}{\hfill\bfseries\Large 插~~~~图\hfill}


\renewcommand{\listalgorithmcfname}{\centering\bfseries\Large 算~~~~法}
\newcommand*{\fullref}[1]{\textbf{\hyperref[{#1}]{第\ref*{#1}章~}}}


%=============================段前段后定义=============================%
\def  \cftbeforetitleskip {35pt}
\def \cftaftertitleskip {25pt}
%% 目录部分
\setlength{\cftbeforetoctitleskip}{\cftbeforetitleskip}
\setlength{\cftaftertoctitleskip}{\cftaftertitleskip}

%% 图目录

\setlength{\cftbeforeloftitleskip}{\cftbeforetitleskip}
\setlength{\cftafterloftitleskip}{\cftaftertitleskip}

%% 表目录
\setlength{\cftbeforelottitleskip}{\cftbeforetitleskip}
\setlength{\cftafterlottitleskip}{\cftaftertitleskip}
%% 算法目录
%\setlength{\cftbeforeloatitleskip}{30pt}
%\setlength{\cftafterloatitleskip}{20pt}




\CTEXsetup[format={\zihao{3}\heiti\centering}]{chapter}
\CTEXsetup[format={\raggedright\zihao{4}\heiti}]{section}
\CTEXsetup[format={\zihao{-4}\heiti}]{subsection}

\CTEXsetup[beforeskip=13pt]{chapter}
\CTEXsetup[afterskip=16.5 pt]{chapter}





\renewcommand{\baselinestretch}{1.5}

\setlength{\baselineskip}{25pt}  %%正文设为25磅行间距


%\CTEXsetup[beforeskip=\cftbeforetitleskip]{chapter}
%\CTEXsetup[afterskip=\cftaftertitleskip]{chapter}





%===============================概念定义===============================%



\newcommand{\citeline}[1]{[\citenum{ #1 }]}



\newcommand{\important}[1]{\CJKunderdot{\textbf{#1}}}

\newcommand{\TheisName}[0]{面向工业界仿冒应用的大规模实证研究}
\newcommand{\ecg}{拓展函数调用图} % 袁宇杰的全局变量。最后可删。
\newcommand{\TheisNameEn}[0]{A Large-Scale Empirical Study on Industrial Fake Apps}

\newcommand{\code}[1]{\textit{ #1 }}
\newcommand{\codeInEqu}[1]{ {  \textit{\kaishu \textbf{#1 }}}}

%\renewcommand{\code}[1]{{ \lstinline{#1}}}

\def\yearOfGrduation{2020} % 毕业年份
\def\monthOfGraduationNum{5} % 毕业月份
\def\monthOfGraduationEng{May} % 毕业月份 (英文)
\def\articleCategory{华东师范大学硕士学位论文} % 论文类型

\def\schoolNameChn{计算机科学与技术学院}
\def\schoolNameEng{School of Computer Science and Technology}


\newcommand{\anonymous}[1]{\phantom{ #1 }} % 匿名开关
\ifdefined \anonymous
  \def\stuID{***********}
  \def\cmajor{*****}
  \def\cfield{*********}
  \def\cadvisor{***~**}
  \def\cname{***}

  \def\emajor{*****}
  \def\efield{*****}
  \def\eadvisor{***}
  \def\ename{***}
\else
  \def\stuID{51174506026}
  \def\cmajor{计算机科学与技术}
  \def\cfield{软件方法与程序语言}
  \def\cadvisor{贺樑~教授}
  \def\cname{唐~崇~斌}

  \def\emajor{\small{Computer Science and Technology}}
  \def\efield{\small{Software Method and Programming Language}}
  \def\eadvisor{\small{Prof.~Liang~He}}
  \def\ename{\small{Chongbin~Tang}}
\fi

\newcommand{\point}[1]{\subsubsection{\textbf{$\S$  #1 }}}
% \renewcommand{\point}[1]{\noindent{\textbf{$\S$  #1 }}}


\newcommand{\Line}[1]{ \textit{ Line .#1}}

\newcommand{\todo}[1]{\textcolor[rgb]{1,0,0}{\textbf{ #1 }}}
\newcommand{\question}[1]{\textcolor[rgb]{1,0,0}{\textbf{ #1 }?}}

\newcommand{\invoke}[2]{  $ (  #1 \to   #2)  $ }
\newcommand{\trigger}[2]{  $ (  #1 \hookrightarrow   #2)  $ }


\newcommand{\methodObjectRel}[3]{  $ (  #1 \stackrel{#2}{\longrightarrow}   #3)  $ }

\newcommand{\algrule}[1][.2pt]{\par\vskip.5\baselineskip\hrule height #1\par\vskip.5\baselineskip}



%%% ----------------------------------------------------------------------



%============================= 版芯控制 ================================%
\setlength{\oddsidemargin}{0.57cm}
\setlength{\evensidemargin}{\oddsidemargin}
\voffset-6mm \textwidth=150mm \textheight=230mm \headwidth=150mm
%\rightmargin=35mm
%                                                                       %


%============================= 页面设置 ================================%
%-------------------- 定义页眉和页脚 使用fancyhdr 宏包 -----------------%
% 定义页眉与正文间双隔线
\newcommand{\makeheadrule}{%
\makebox[0pt][l]{\rule[.7\baselineskip]{\headwidth}{0.4pt}}%
\rule[0.85\baselineskip]{\headwidth}{0.4pt} \vskip-.8\baselineskip}
\makeatletter
\renewcommand{\headrule}{%
{\if@fancyplain\let\headrulewidth\plainheadrulewidth\fi
\makeheadrule}} \makeatother

\newcommand{\adots}{\mathinner{\mkern 2mu%
\raisebox{0.1em}{.}\mkern 2mu\raisebox{0.4em}{.}%
\mkern2mu\raisebox{0.7em}{.}\mkern 1mu}}

\setmainfont{Times New Roman}
\dottedcontents{chapter}[1.5cm]{\xiaosi\heiti}{3.8em}{9.5pt}
\lhead{\articleCategory}

\dottedcontents{section}[1.5cm]{\xiaosi\heiti}{2.8em}{9.5pt}
\lhead{\articleCategory}

% \setlength{\parindent}{2em}
% %默认的弹性间距会导致文中某些排版flush的时候,出现大量空白。
% \setlength{\parskip}{0.5em} %指定固定段后间距,默认为弹性间距。
% \setlength{\intextsep}{10pt} %固定浮浮动体前后间距。

%=============================== 代码格式定义 ================================%


%% 辅助功能显示页边界
% \usepackage{showframe} % just for the example

\usepackage{etoolbox}



\input{settings-lst}




%=============================== 正文部分 ================================%

\begin{document}
\input{A1-COVER-1.tex}
\clearPaperPage



\pagestyle{empty}

% \begin{flushleft}
% 	\large
% 	\renewcommand\arraystretch{1.5}
% 	\begin{tabular}{l}
% 		\noindent{\large Dissertation for master degree in \yearOfGrduation}  \\
% 		\noindent{\large  }\\
% 	\end{tabular}
% % \hskip 1 cm
% \renewcommand\arraystretch{1.5}
% \begin{tabular}{lc}
% 	 University Code:  &  10269  \\
%  	 Student ID: &    \stuID  \\
% 	% \noindent{{\zihao{4} 学\qquad 号:\underline{\anonymous{51164500190}{ *** }}}}\\
% 	\end{tabular}
% \end{flushleft}

\begin{flushleft}
\hspace{-0.5cm}
\renewcommand\arraystretch{1.5}
\begin{tabular}{l}
\noindent{\large Thesis for master's degree in \yearOfGrduation}  \\
\noindent{\large  }\\
\end{tabular}
\hskip 0.5 cm
% \renewcommand\arraystretch{1.5}
\begin{tabular}{lc}
\noindent{{\zihao{4} University Code: }} & 10269\\
\noindent{{\zihao{4} Student ID: }} & ~~~\stuID~~~\\
%\noindent{{\zihao{4} 学\qquad 号:\underline{\anonymous{51174506026}{ *** }}}}\\
\end{tabular}
\end{flushleft}

\vskip 2cm

\begin{center}
{\Huge $\mathbf{EAST}\,\mathbf{CHINA}\,\mathbf{NORMAL}\,
\mathbf{UNIVERSITY}$}
\end{center}

\vskip 3cm

\begin{center}
\bfseries{\scshape{\huge \TheisNameEn
}}\\
\end{center}

\vskip 2cm {\large
\begin{center}
\begin{tabular}{l}
Department:\\
Major:\\
Research Direction:\\
Supervisor:\\
Candidate:
\end{tabular}
\begin{tabular}c
\small \schoolNameEng \\
\hline \emajor  \\
\hline \efield\\
\hline \eadvisor\\
\hline \ename\\
\hline
\end{tabular}
\end{center}}

\vskip 30mm

\begin{center}
{\Large \monthOfGraduationEng, \yearOfGrduation}
\end{center}

\clearPaperPage

\input{A3-COPYRIGHT.tex}
\clearPaperPage

\pagestyle{empty}
$$\\ \\ \\ $$

\centerline{\bf\Large $\underline{\mbox{\cname}}$硕士学位论文答辩委员会成员名单}

\vskip 10mm

\ifdefined \anonymous
	\def\nameProfA{***}
	\def\titleProfA{***}
	\def\nameProfB{***}
	\def\titleProfB{***}
	\def\nameProfC{***}
	\def\titleProfC{***}
	\def\affiliationA{******}
	\def\affiliationB{******}
	\def\affiliationC{******}
\else
	\def\nameProfA{柳银萍}
	\def\titleProfA{教授}
	\def\nameProfB{	章炯民}
	\def\titleProfB{副教授}
	\def\nameProfC{谢瑾奎}
	\def\titleProfC{副教授}
	\def\affiliationA{华东师范大学数学科学学院}
	\def\affiliationB{华东师范大学计算机科学与技术学院}
	\def\affiliationC{华东师范大学计算机科学与技术学院}
\fi

\begin{center}\large
	\begin{tabular}{ |p{25mm} < {\centering}|p{30mm} < {\centering}|p{48mm} < {\centering}|p{25mm} < {\centering}| }
		\hline
		\heiti  姓名 &\heiti  职称&\heiti  单位&\heiti  备注 \\
		\hline
		\nameProfA & \titleProfA & \affiliationA &  主席 \\
		\hline
		\nameProfB & \titleProfB & \affiliationB & \\
		\hline
		\nameProfC & \titleProfC & \affiliationC & \\
		\hline
	\end{tabular}
\end{center}

\clearPaperPage

\pagenumbering{roman}
\pagestyle{plain}

\addcontentsline{toc}{chapter}{摘要}

\chapter*{\zihao{2}\heiti{摘~~~~要}}
\vspace{-5mm}

\setlength{\baselineskip}{25pt} % 25磅行距

作为市场占有率最高的智能手机操作系统,安卓系统拥有基数庞大的用户群体,也吸引了无数开发者为其开发应用,构筑出一个生机勃勃的生态系统。
然而,在浩如烟海的安卓应用中,潜藏着形形色色的移动黑灰色产业链条,其中既包括如恶意应用一类的研究热点,也有仿冒应用等鲜受关注的领域。
有别于官方发布的正版应用,仿冒应用属于移动灰色产业的一环,其目的各异,难以一概而论。
而与以恶意应用为代表的研究热点不一样,我们对仿冒应用的生态并不了解,对其行为、特征更是知之甚少。

得益于对恶意应用较为充分的理解,业界对恶意应用的监控成为了可能。
因而即使爆发了新型的恶意应用,各厂家也能及时推出具有针对性的方案。
相对地,我们对仿冒应用的认识匮乏,则很可能会成为隐患。
仿冒应用都有怎样的形态?仿冒应用在各个市场上的分布如何?它们是否会包含恶意行为?有什么样的发展趋势?
这些问题都尚未有人给出过解答。
在未有前人研究基础的情况下,直接从业界收集数据入手分析无疑是获得第一手资料的最佳途径。
因此,我们进行了针对仿冒应用的大规模数据收集和实证研究,并且结合市场上的评论对仿冒应用获得的反馈作出了进一步分析。

在实证研究方面,针对上述问题,我们从三个不同视角对数据进行了探究,其分别为:仿冒应用的基本特征,影响仿冒应用数量的因素,以及仿冒应用的发展轨迹。
三个视角由浅入深,从仿冒应用的应用名、包名和APK包大小等基本信息特征开始测量,再对可能与仿冒应用数量关联的因素进行量化分析,最后引入时间因素对数据进行挖掘。
从三个不同视角的分析中,本文提供了包括仿冒应用命名倾向、仿冒应用作者对应用市场拦截的规避策略等珍贵的领域知识。
本文还对数据中较为特别的样本作出了详尽的案例分析,除了可以印证上述的领域知识与发现之外,也能引起我们对现今应用市场生态环境的思考。

在用户反馈分析方面,我们针对仿冒应用在应用市场上获得的评级和评论等进行了一系列的分析与验证。
鉴于前人研究中有对应用进行排名欺诈行为的探索,我们也使用了两个不同的方法,从不同方面验证我们寻找到的仿冒应用中是否存在排名欺诈行为。
结合最后的人工复核,我们确认了仿冒应用中确实存在排名欺诈行为的事实,向仿冒应用生态认知的谜题补上了一块拼图。

借助本文提供的数据和分析结果,我们希望读者能获得一个面向仿冒应用及其生态的清晰视角。
同时,我们更希望本文能抛砖引玉,吸引更多科研人员投入到对移动灰色产业的观察研究中。

\sihao{\heiti{ 关键词:}} \xiaosi{Android应用程序, 仿冒应用, 实证研究, 数据分析, 排名欺诈}

\clearPaperPage

\newpage

\addcontentsline{toc}{chapter}{ABSTRACT}

\chapter*{\zihao{2}\heiti{ABSTRACT}}
\vspace{-5mm}

% 作为市场占有率最高的智能手机操作系统,安卓系统拥有基数庞大的用户群体,也吸引了无数开发者为其开发应用,构筑出一个生机勃勃的生态系统。
As the smartphone OS with the highest market share, Android owns countless users and has attracted numerous developers to develop apps on it, building a thriving application ecosystem.
% 然而,在浩如烟海的安卓应用中,潜藏着形形色色的移动黑灰色产业链条,其中既包括如恶意应用一类的研究热点,也有仿冒应用等鲜受关注的领域。
However, underlying the uncountable number of apps are all sorts of mobile underground profit chains.
Not only malware, one of the current study focuses, but also fake apps, an area we barely keep eyes on, are included in these chains.
% 有别于官方发布的正版应用,仿冒应用属于移动灰色产业的一环,其目的各异,难以一概而论。
Different from official apps published by legal developers, fake apps belong to the mobile underground industry.
Their purposes are various, which is difficult to generalize.
% 而与以恶意应用为代表的研究热点不一样,我们对仿冒应用的生态并不了解,对其行为、特征更是知之甚少。
Unlike what we have known about malware, our domain knowledge on fake apps' ecosystem is almost none, let alone the understanding of their behavior or characteristics.

% 得益于对恶意应用较为充分的理解,业界对恶意应用的监控成为了可能。
Thanks to our full understanding of malware, it is now possible to monitor malware on the industrial level.
% 因而即使爆发了新型的恶意应用,各厂家也能及时推出具有针对性的方案。
Therefore, even if there is an outbreak of malware in the new form, manufactures are able to propose remedies timely.
% 相对地,我们对仿冒应用的认识匮乏,则很可能会成为隐患。
Comparatively, our lack of fake apps' knowledge is very possible to become a snake in the grass.
% 仿冒应用都有怎样的形态?仿冒应用在各个市场上的分布如何?它们是否会包含恶意行为?有什么样的发展趋势?和其他黑灰产环节是否会有联系?
What is the form of fake apps? How are they distributed in different app markets? Do they perform malicious behavior? What kind of developing tendency do they have? Are they connected to other branches of the mobile underground industry?
% 这些问题都尚未有人给出过解答。
None of these questions were answered so far.
% 在未有前人研究基础的情况下,直接从业界收集数据入手分析无疑是获得第一手资料的最佳途径。
Without any previous study, there is no doubt that collecting and analyzing data directly from the industry is the best way to win the first-hand information.
% 因此,我们进行了针对仿冒应用的大规模数据收集和实证研究,其中实证研究分为数据挖掘,案例分析和用户反馈分析三个部分。
Thus, we employ a large-scale data gathering and empirical study towards fake apps, in which the empirical study part is consists of three parts: data mining, case study, and user feedback analysis.

% 在数据挖掘方面,针对前面提到的问题,我们从三个不同视角对数据进行了探究,其分别为:仿冒应用的基本特征,影响仿冒应用数量的因素,以及仿冒应用的发展轨迹。
In terms of data mining, we leverage three different perspectives to explore our data in order to answer the aforementioned questions.
% 三个视角由浅入深,从仿冒应用的应用名、包名和APK包大小等基本信息特征开始测量,再对可能与仿冒应用数量关联的因素进行量化分析,最后引入时间因素对数据进行挖掘。
The three perspectives are, namely, fake apps' characteristics, factors affecting fake apps' number, the developing trends of fake apps.
They follow an easy-to-complex pattern.
We start from the basic pattern of fake apps, then perform quantitative studies on factors which are possible to affect the number of fake apps, and lastly introduce time factor to mine the data.
% 从三个不同视角的分析中,本文提供了包括仿冒应用命名倾向、仿冒应用作者对应用市场拦截的规避策略等珍贵的领域知识。
As a result, the three perspectives provide us with valuable domain knowledge, like fake apps’ naming tendencies and fake developers’ evasive strategies.

% 案例分析方面,我们从收集到的数据中手动筛选出了其中具有代表性的三个案例。
And then, on the case analysis's side, we manually screen out the most representitive cases from our data.
% 这些案例除了可以印证上述的领域知识与发现之外,还提供了更多关于仿冒应用生态的细节,值得引起我们对现今应用市场生态环境的思考。
These cases can, not only confirm the findings and domain knowledge mentioned above but also provide more details about the nature of fake apps, raising our concern on the app markets' status-quo.

% 在用户反馈分析方面,我们针对仿冒应用在应用市场上获得的评级和评论等进行了一系列的分析,以了解用户对仿冒应用的态度及验证仿冒应用与移动黑灰产中的排名欺诈是否存在关联。
Last but not least, on feedback analysis, we conduct a series of analyses on the ratings and comments the fake apps earned from an app market to answer two questions: How do users think about fake apps? Will fake apps cooperate with ranking fraud --- another branch of the mobile underground industry?
% 鉴于前人研究中有对应用进行排名欺诈行为的探索,我们也使用了两个不同的方法,从不同方面验证我们寻找到的仿冒应用中是否存在排名欺诈行为。
Due to the fact that previous studies revealed the existence of ranking fraud in App markets, we also utilize two different measurements to verify whether the fake apps leverage ranking fraud service.
% 结合最后的人工复核,我们确认了仿冒应用中确实存在排名欺诈行为的事实,向仿冒应用生态认知的谜题补上了又一块拼图。
As discerned and confirmed by our manual review, we make sure that some fake apps do deploy ranking fraud services to raise their exposure.
By revealing this, we gather another piece of evidence to implement the puzzle of fake apps' nature, making our study more complete.

% 借助本文提供的数据和分析结果,我们希望读者能获得一个面向仿冒应用及其生态的清晰视角。
Through the data and results provided by this paper, we hold an expectation that our readers obtain a clear vision of fake apps and their ecosystem.
% 同时,我们更希望本文能抛砖引玉,吸引更多科研人员投入到对移动灰色产业的观察研究中。
Meanwhile, we also long for raising more researchers' interests and participation, shedding more light on the mobile underground industry.

% \sihao{\heiti{ 关键词:}} \xiaosi{Android应用程序, 仿冒应用, 实证研究, 数据分析, 排名欺诈}
{\sihao{\textbf{\newline Keywords:}}} \textit{Android application, Fake App, Empirical Study, Data Analysis, Ranking Fraud}

\clearPaperPage

%\clearpage{\pagestyle{plain}\clearPaperPage}
%\textcolor[rgb]{1,1,1}{文本}

\input{B2-TOC.tex}
\clearPaperPage

\input{B3-Figure.tex}
\clearPaperPage

\input{B4-Table.tex}
\clearPaperPage
%%%%%%%%%%%%%%%%%%%%%%%%%%%%%%%%%%%%%%%%%%%
%                 算  法                  %
%%%%%%%%%%%%%%%%%%%%%%%%%%%%%%%%%%%%%%%%%%

\makeatletter
\renewcommand*{\l@algocf}[2]{\@dottedtocline{1}{1em}{2.3em}{算法 #1 }{#2}}
\makeatother

\listofalgorithms

%\clearPaperPage

\pagenumbering{arabic}
\pagestyle{fancy}
\fancyhead[LE,RO]{\small{\articleCategory}}
\fancyhead[RE,LO]{ \small\leftmark}

% 正文
% 在章首页添加页眉

\fancypagestyle{plain}{%
\fancyhead[LE,RO]{\small{\articleCategory}}
\fancyhead[RE,LO]{ \small\leftmark}
}

% \banxiaosi
\xiaosi

\clearPaperPage


\chapter {绪论}
\label{chp:intro}

\section{选题背景}

% With the growing attention of mobile markets, Android has accounted for 85.9\% of global market share~\cite{Gartner_report}. Over 1.5 million apps were released within 2017 alone.
% Along with the booming of Android markets is the flourish of the mobile underground industry. \emph{Fake apps}, i.e., apps without official certificates, account for a major part of such underground industry.
% Specifically, we consider fake apps as those who simulate the corresponding official ones or look almost the same as their official correspondences, with ultimate goal to elicit download or manifest malicious behaviors.
% Early observation reveals fake apps come in two different forms.
% The first category is called \texttt{imitators}, a group of apps with similar names or functionalities to their official correspondences so that users are fooled to download them.
% While imitators are just similar to official apps, \texttt{imposters}~\cite{Andow2016ASO} refer to the category of apps that have exactly the same metadata with their official correspondences, for example, they may have the same names, icons, or version numbers, some of the imposters are even made by repackaging official apk files directly.
随着移动市场于近年逐渐兴起,Android系统作为一个主流的移动端操作系统也在蓬勃发展。
数据分析机构StatCounter资料显示,Android市场占有率自发布之日起就在逐年稳步增长。
截至2020年,Android系统已经占据全球移动端市场份额的74.3\%~\cite{MobileOSMktShare}。
与此同时,Android应用的数量也伴随着Android市场的蓬勃发展节节攀高。
仅Android官方的应用商店Google Play就在2017年中新上架了近一百万个可供下载的应用程序。
虽然因为各种原因,Google Play上的应用数量在2018年有所回落,但如\autoref{fig:app_number}所示,应用市场上目前仍有近三百万个可用的应用程序,Android应用市场依然充满活力~\cite{StatistaAppNumber}。

\begin{figure}[htbp]
	\centering
	\includegraphics[width=0.9\textwidth]{./Figures/edwin-intro-app-number.png}
	\caption{Google Play 应用商店架上应用总数变化趋势}
	\label{fig:app_number}
\end{figure}

伴随着应用数量爆发式增长的,还有欣欣向荣的移动黑灰产。
黑灰产是以侵害用户、原应用作者或其他第三方利益为手段,凭此或通过其他可疑方式牟利的产业。
现阶段已知的移动黑灰产包括恶意应用的编写与传播产业、重打包应用的制作产业和在应用市场上提供批量虚假评价的产业。
据上述定义,\emph{仿冒应用}也是移动黑灰产业链中的一环。
本文提及的仿冒应用指代模仿市面上热门的应用、甚至外观与热门应用相差无几的移动应用,其目的是诱导用户下载以赚取流量,甚至是窃取用户信息、触发有害行为从而盈利。
根据的前期观察,本文发现仿冒应用以两种不同的形式出现。
第一类为\texttt{模仿应用},这类应用具有和原版热门应用相似的外观(比如名字、图标等),诱导用户下载。
而第二类---\texttt{高仿应用}~\cite{Andow2016ASO, luo2016repackage},已不单单是与原应用``相似''了,这类应用采用和原应用一模一样的外观,乃至连版本号都相同,其中的部分应用就是直接通过对原应用重打包做成的。

% Such fake apps pose significant problems to not only the official developers' interest but also the end users' right.
% For example, when users try to search an app for installation in market, multiple fake apps with similar names or icons will be retrieved at the same time.
% As a result, the user experience of app searching and downloading is greatly affected by the fake apps in real world.
这些仿冒应用不仅大大损害了原应用开发者的利益,也侵犯了用户的权益。
可以假设一个普遍场景:当用户尝试通过应用市场搜索安装一个应用,应用市场往往会返回多个无论是名字或是图标都十分相像的结果。在未有明确指引的情况下,用户十分有可能安装一个仿冒应用。
即使用户最后删除了仿冒应用,并重新装上原版应用,也浪费了时间和人力成本。
而前人研究~\cite{Zhou2012DissectingAM}显示,应用中的某些恶意行为可被自动触发。
万一用户下载的仿冒应用中包含此类恶意行为,将会对用户造成更大的损害。
可见,仿冒应用严重影响用户搜索与下载应用时的安全和体验。

% Even worse, as the doorsill to develop an app has been set low, the cost to develop a fake app is much lower than what it takes to develop a desktop program, providing an ideal hotbed for the underground industry to thrive on~\cite{wasserman2010software}. Moreover, the flexibility of Android app implementation~\cite{storydroid} contributes the fake apps' complexity.
一方面,随着开发移动应用的门槛逐渐下降,开发一个仿冒应用的成本已经远低于开发一个桌面级应用所需的成本,为地下产业涌入移动端发展提供了绝佳的温床~\cite{wasserman2010software}。
另一方面,移动应用功能在实现上的灵活性~\cite{storydroid}也增加了仿冒应用的复杂度,让分析移动应用变得更加困难。

\section{研究现状}
尽管如今仿冒应用随处可见,业界和学术界对仿冒应用和他们的生态却依然知之甚少——他们有何共同特征、数量多少、迭代速度如何、以及他们如何规避应用市场检测等问题依然有待解答。
据本文的前期调研,目前还未有任何关于仿冒应用或其生态系统的研究。
因此,本节将介绍现有针对其他移动黑灰产业的相关研究,分析各产业或研究与仿冒应用的关联。

% \subsection{Empirical study on grayware}
\subsection{针对灰色应用的实证研究}
% Andow et al.~\cite{Andow2016ASO} proposed a study of grayware, in which 9 types of greyware are defined and triaged from data retrieved from google play. We referred the definition on \textit{imposter} from this article.
Andow等人针对灰色应用的研究~\cite{Andow2016ASO}从Google Play应用商店中采集了多个应用样本。
该研究将样本分类,定义出了9种不同的灰色应用。
灰色应用指并非具有明显的恶意行为,但应用意图存疑、又或是会向系统申请过多权限的应用程序。
灰色应用不是恶意应用,但由于其盈利方式可疑,可将其归入移动黑灰产内。
本文中对\texttt{高仿应用}的定义参照了该文献的内容。

% \subsection{Empirical studies on malware ecosystem}
\subsection{针对恶意应用生态系统的研究}
% 46 malware samples on various platforms are dissected to gain understanding on their incentive system as in a survey conducted by Felt et al.~\cite{Felt2011ASO}. Meanwhile, several strategies are proposed by them to defend again these type of malware.
% Zhou and Jiang~\cite{Zhou2012DissectingAM} gathered over 1,200 malware samples across major Android malware families, systematically characterizing their different natures including installation methods, activation mechanisms and how the payload is carried out.
% These researches help expand practitioners' horizon in terms of malicious app's behavior, but the insight they provide may not suit fake app identification well.
恶意程序自桌面时代起直到如今的移动互联网时代,一直是软工与安全领域的研究重点。
在Felt主导的一次研究~\cite{Felt2011ASO}中,研究人员仔细剖析了来自多个不同平台的46个恶意程序样本以了解这些样本的激励机制。该篇文献也揭示了这些样本的运行机制和行为策略,为后人抵御此类恶意行为提供参考。
另外,Zhou和Jiang~\cite{Zhou2012DissectingAM}搜集了来自多个主要恶意应用家族的、超过1,200个恶意应用样本,并系统性地描绘了这批样本的不同性质,包括其安装手段、激活机制和其如何执行有效负载(实现恶意行为);在另一篇著作~\cite{zhou2012hey}中,他们提出了名为DroidRanger的系统,成功地从5个应用市场的204,040个应用中找出了211个恶意应用。

这类研究帮助了业界人员拓宽视野,使得业界人员对恶意应用的行为更加了解。
然而根据前文定义,仿冒应用与恶意应用并不相同,针对仿冒应用的专门研究依然有必要。

% \subsection{Repackage detection}
\subsection{针对重打包应用检测研究}
\label{sec:repackaging}
% Prior work on repackage detection generally falls into five categories.
% The first one is based on apps' \textit{instruction sequences}, which uses fuzzy hashing techniques to extract the digest of apps, then calculates similarity between every two digests~\cite{DroidMOSS,Zheng2013DroidAnalyticsA}.
% The second one is based on \textit{semantic information}.
% CLANdroid~\cite{CLANdroid} detects similar apps through analyzing five semantic anchors (e.g., identifiers and Android APIs).
% The third kind leverages \textit{lib detection} methods.
% CodeMatch~\cite{CodeMatch} filters out libraries used in apps then compares the hash of their remnant.
% Wukong~\cite{Wukong} detects repackage apps in two steps, but that it processes the second step by using a counting-based code clone detection approach, instead of hash.
% ViewDroid~\cite{ViewDroid} picks out repackage apps by rebuilding and comparing the viewgraph of different apps, belongs to the forth kind which makes use of \textit{visualizes information}.
% The fifth kind applies \textit{graph theory} on measuring app similarity.
% DNADroid~\cite{DNADroid} calculate apps' similarity based on program dependency graph (PDG), while AnDarwin~\cite{AnDarwin} builds semantic vectors with PDG extracted from every methods.
% Centroid~\cite{Centroid} even constructs 3D-control-flow-graph (3D-CFG) for each method in an app and see how alike the centroid in different apps are.
重打包指恶意开发者对原应用解包、篡改内容之后再将应用重新打包的技术。
重打包应用侵害了应用原作者的知识产权,因此也属于移动黑灰产业范畴。
针对重打包检测的前人研究大致可划分为五个类别:

第一类基于应用\textit{指令序列}。这类方法使用模糊哈希的方法提取出应用的摘要信息,通过比对两两应用之间的摘要信息获得应用之间的相似度~\cite{DroidMOSS, Zheng2013DroidAnalyticsA}。
第二类凭借\textit{语义信息}比对应用。CLANdroid~\cite{CLANdroid}通过分析五种语义特征点(比如代码中的标识符和调用到的Android API等),以检测相似应用。
第三类利用\textit{第三方库}进行检测。CodeMatch~\cite{CodeMatch}筛选出应用中使用的第三方库代码后,计算并比对剩余部分代码的哈希值,获取不同应用的相似程度。
Wukong~\cite{Wukong}也分两步检测重打包应用,但与CodeMatch相比,其第二步使用了基于计数的代码克隆检测手段,而非基于哈希的技术。
ViewDroid~\cite{ViewDroid}通过重建和比对应用的视图来筛出重打包应用,属于第四类---\textit{信息可视化}。
第五类依赖\textit{图论}衡量应用相似性。
DNADroid~\cite{DNADroid}基于应用的程序依赖图(Program Dependency Graph, PDG)比对应用,而AnDarwin~\cite{AnDarwin}则用从每个方法中提取的PDG构建出语义向量,再计算向量间相似度以检测重打包应用。
Centroid~\cite{Centroid}甚至为应用中的每个函数构建了三维控制流图(3-Dimensional Control Flow Graph, 3D-CFG),再将三维控制流图聚合,通过检测不同应用在控制图聚合后的质心位置判断应用间的相似程度。

% Each of these approaches has its own advantages and drawbacks, from the perspective of scalability and accuracy, which are beyond the topic in this article.
% One common they all do share, however, are that the verification step, without any exception, is based on certificate system.
% Once the illegal developers poison data with legal certificates through apps with vulnerable signature scheme, even the state of the art detecting approach can do nothing about it.
以上检测方法各有其优缺点,但在最后的验证阶段都需要使用Android安全证书对APK文件进行验证,以确定是否误判。
因此,在原版Android安全证书已知的前提下,直接用证书比对是最简洁有用的验证方式。

虽然重打包应用与仿冒应用互有重合之处,但并非所有仿冒应用都为重打包应用。
因此,要了解仿冒应用的性质和特征,只对重打包应用进行观察是不足的,需要详尽地从应用市场上收集样本后再进行探索。

\subsection{针对应用评论相关移动黑灰产的研究}
近年来,除了利用应用程序牟利以外,移动黑灰产业也开始进入应用市场评论的领域。
大量的前人工作~\cite{hernandez2019the, xie2014grouptie, zhu2014discovery, hu2019want, chen2017toward, xie2016you, hooi2016fraudar}揭示,很多应用市场都饱受虚假评论的困扰。
Rahman等人在2019年发布的关于虚假评论行为的实证研究~\cite{rahman2019art}在公开确认了这个产业链存在的同时,也揭露了该产业的行为模式、生存情况甚至是从业人员收入水平等方面的信息。
作为黑灰色产业链下游环节,虚假应用评论(即排名欺诈行为)很可能与仿冒应用相结合,为移动黑灰产从业者牟取更多利益。

\section{问题分析与研究难点}

前文的研究现状,不仅反映出移动黑灰产业是目前的研究热点之一,更反映出移动黑灰产从业人员无孔不入的特点。
为了保障正当开发者的利益与消费者的权益,学术界和工业界都需要对移动黑灰产有更全面、更深入的理解,从而更好地预防未知的威胁。
上述研究现状表明,工业界与学术界的软工、安全领域在移动黑灰产的研究上已经取得了丰富的成果,然而现有研究提供的知识仍有空缺部分,仿冒应用部分正是缺口之一。
现阶段,在缺乏仿冒应用相关研究的情况下,仿冒应用数据缺失,造成了以下问题:

1)\	\emph{无法对仿冒应用进行定量分析} \quad
目前,学术界与工业界目前对恶意应用和排名欺诈行为均有良好理解,厂家得以在实践中抵御、规避此类移动黑灰产的侵袭,这得益于前人在实证调查研究~\cite{Felt2011ASO, Zhou2012DissectingAM, zhou2012hey, rahman2019art}中提供的数据与洞见(Insight)。
然而,在仿冒应用数据缺失的情况下,相关定量分析与定性分析无法进行,无法让学术界与工业界对仿冒应用有清楚了解。
针对仿冒应用进行实证研究可以有效解决这个问题。
具体而言,可利用实证研究中的混合方法途径(Mixed-methods approaches)方法论~\cite{easterbrook2008selecting},先从工业界环境(即各应用市场)中获取大量应用,再从所获应用中进行筛选,可获得一定数量的仿冒应用,从而开展相关定量分析。

2)\	\emph{无法确定仿冒应用的性质} \quad
尽管仿冒应用在日常生活中随处可见,却少有研究者对其进行研究。
因而,仿冒应用的形态尚未明确,亦从未有前人评估仿冒应用的风险性;总结仿冒应用的发展趋势不明,无法确定这一移动黑灰产是否会更加壮大;仿冒应用的市场反馈无人探索,仿冒应用是否会与其他黑灰产业环节结合更是不得而知。
针对以上疑问,可采用实证研究的混合方法途径方法论可以通过对数据定性分析,帮助理解仿冒应用具有的性质;而相对的案例分析(Case studies)方法论~\cite{easterbrook2008selecting}则可以在案例支持下更有力地确认定性分析的发现。

综上,现时仿冒应用数据匮乏、对仿冒应用了解缺失的问题,可通过实证研究缓解。
因此,本文与犇众信息的移动安全威胁数据平台Janus~\cite{janus}合作,搜集并分析了大量应用样本,从仿冒应用特征解读和仿冒应用评论分析两方面进行了大规模实证研究,对仿冒应用作出了较为全面的剖析,填补了本领域的研究空白。
仿冒应用特征解读利用数据挖掘分析技巧,利用本研究收集到的仿冒应用数据对仿冒应用进行画像;
仿冒应用评论分析则通过收集仿冒应用在市场上获得的反馈,验证仿冒应用和排名欺诈行为的关系,进一步地提供了关于仿冒应用生态的信息。

在数据挖掘分析中,研究者通常都会遇到几点挑战:如何确定研究主体、如何收集数据、如何对数据进行有效处理;
而在排名欺诈相关研究中,如何从评论数据中有效挖掘排名欺诈行为是研究者经常要思考的问题。
故本实证研究中的难点可概括如下:

1)\	\emph{如何确定仿冒应用} \quad
仿冒应用和正版应用是相对的概念。
本文选择了市面上最热门的50个应用,再搜集其对应的仿冒应用样本。
从应用中筛选出与热门应用外观相似或是相同的样本后,本文使用Android本身自带的证书机制,将获得样本的证书信息与原版应用的证书信息进行比对,从而鉴别出仿冒的样本。

2)\	\emph{如何获得针对仿冒应用的大量数据} \quad
数据搜集是科研工作中公认的难点。
本文想要提供一次全面的研究结果,除了搜集的目标应用需要有多样性之外,也必须获得不同应用市场上的数据,增加研究的代表性。
前文提及犇众信息的移动安全威胁数据平台Janus是一个数据整合平台。该平台每天从各大Android应用市场爬取应用样本入库,免去了要针对各个市场重新定制爬虫代码的麻烦。
通过设计和实现仿冒应用收集框架\mytool,本文顺利从Janus搜寻到了近14万条数据条目作为原始数据,其中每条数据条目代表Janus从应用市场上获得的一个应用样本。

3)\	\emph{如何对大量的数据进行有效处理} \quad
数据规模和处理效率一直是一对矛盾。
由于一条数据条目代表一个应用样本,要对所有应用样本进行详尽分析,明显太耗费时间成本与计算成本;然而,如果只对样本进行简单处理,获得的分析结果就不够全面和深入。
在尽量确保分析全面性的前提下,对于每个样本,本研究只抽取8个关键信息项进行分析,以节省时间与计算成本。

4)\ \emph{如何挖掘评论中的排名欺诈行为} \quad
现有的排名欺诈挖掘工作均具有其各自的局限性,或是对数据有连续收集要求,或是要求检测者有已知的排名欺诈群体,并不能直接应用到本研究收集的数据中。
为此,本研究先后设计了基于用户行为可信度和基于评论内容重复度的两个方法。
前者规避了现有方法的局限性,后者进一步解决了大数据量带来的大运算量问题。

\section{研究方法与工作概览}

\subsection{研究方法}
前文已经提及,实证研究能够为研究主体提供画像与洞见,从而为对应方面的后续实践与研究提供建议与便利。
在软件工程领域与安全领域,已有多篇实证研究为实践工作~\cite{Felt2011ASO, Zhou2012DissectingAM, zhou2012hey, rahman2019art, wu2016ji, yang2015xin}提供知识支持。
因此,本文亦将采用实证研究方式,对仿冒应用进行探索。

实证研究是一种基本研究手段,旨在针对技术在实际应用场景下的真实状况或对应产物进行数据收集、调查与分析。
Easterbrook等人于2008年提出了面向软件工程领域的实证研究方法建议\cite{easterbrook2008selecting},为软工实证研究提供方法论指导,其中将实证研究方法论分为受控实验、案例分析、调查研究、社会学意义研究与混合方法途径等多类,分别适用于不同场景。
本文研究中使用到的是混合方法途径与案例分析,简介如下:

1)\ \emph{混合方法途径(Mixed-method approaches)} \quad
混合方法途径指结合定量分析与定性分析,对研究对象进行系统数据解读的实证研究方法。
按照实施的方式,混合方法途径可分为顺序解释策略(Sequential explanatory strategy)、顺序探索策略(Sequential exploratory strategy)与并发三角策略(Concurrent triangular strategy)三类。
其中,顺序解释策略先收集与分析定量数据,再收集和分析定性数据,以定性数据结果帮助解释定量结果;与之相反,顺序探索策略先收集与分析定性数据,再收集和分析定量数据,以定量数据结果帮助解释定性结果;并发三角策略则会同时采用不同方法,以试图确认、交叉验证或证实已有发现。
本文的特征解读部分先采用定量分析方法分析数据,再采用案例分析方法给出定性结果,使用了顺序解释策略;而后续的仿冒应用与排名欺诈关联验证先给出定性结论,再提供定量数据支持,则使用了顺序探索策略。

2)\ \emph{案例分析(Case studies)} \quad
案例分析是软件工程领域最常用的实证研究方法,完整的案例分析通过确立研究问题、选择研究案例和收集数据三步研究真实场景中出现的现象,适用于真实环境为对研究主体产生影响的因素之一、又或是实验数据时间跨度较大的场景。
对于针对某些现象的初步调查,可使用探索性案例分析(Exploratory case studies)以提出新猜想和构建理论;而验证性案例分析(Confirmatory case studies)则用于验证现存理论。
本文的案例分析既包含用于提出新猜想的探索性案例分析,亦包含验证数据画像的验证性案例分析。

\subsection{工作概览}

\begin{figure}[htbp]
	\centering
	\includegraphics[width=\textwidth]{./Figures/edwin-overview}
	\caption{本文工作概览}
	\label{fig:Workflow}
	\vspace{-3mm}
\end{figure}

本节为本文工作提供概览。如\autoref{fig:Workflow}所示,本工作通过三个主要部分完成实证研究:

1)\ \emph{面向仿冒应用的收集框架\mytool } \quad
针对现有爬虫框架不能定向爬取应用的问题,本研究设计并实现了仿冒应用收集框架\mytool,以进行仿冒应用的数据收集。
数据收集主要分为两个部分:正版应用信息的收集和仿冒应用的收集。
在正版应用信息收集的部分,本文选择了50个最热门的App作为目标应用,然后手动收集了跟这些App有关的信息;
仿冒应用信息收集方面,本研究利用\mytool,通过Janus平台收集了从各个应用商店获得的大量仿冒应用样本。

2)\ \emph{结合案例分析的仿冒应用特征解读} \quad
针对移动黑灰产研究中关于仿冒应用的研究空白,本研究收集仿冒应用数据,结合案例分析,进行了首次基于Android系统仿冒应用的特征解读。
特征解读从三个视角完成,这些视角分别是仿冒的基本应用特征、影响仿冒应用数量的因素和仿冒应用的发展轨迹,由浅入深揭示仿冒应用的生态。
对应的三个案例分析除了为特征解读提供案例支持之外,还揭示了更多仿冒应用开发者的行为特征。

3)\ \emph{面向仿冒应用的排名欺诈验证方法} \quad
在这个部分,本文从第三方应用市场中随机选取一部分应用,爬取了用户对它们的所有历史评价,以检测仿冒应用与排名欺诈行为的关联。
针对前人检测研究需要先验知识或特殊数据的问题,本研究从社交媒体研究引入了用户行为可信度进行排查,避开了现有方法的局限性。
进一步地,本研究从评论内容重复率方面提出了另一创新性排查方法,解决了数据量增大带来的大运算量问题。
人工复查结果显示,两种排查方法均取得了优于现有方法的结果。


\section{本文组织结构}
本文共分为五章,环绕着本研究的数据搜集、分析过程和分析结果展开,各章节内容如下:

\fullref{chp:intro} 主要提供了本文的研究背景、相关工作、研究意义和工作概览。

\fullref{chp:background} 介绍了Android应用的构建流程、Android安全证书机制、应用市场与黑灰产业的关系和一些移动黑灰产知识。

\fullref{chp:fakerevealer} 阐述了仿冒应用收集框架\mytool 的设计缘由,说明了仿冒应用的数据收集方式并详细介绍了其中三个组件(\componentA、\componentB 和\componentC)的设计与实现,最后提供仿冒应用数据概览。

\fullref{chp:discoveries} 从多个视角分类提出并解说针对这批仿冒应用数据得到的发现。
这些视角包括仿冒应用特征、影响仿冒应用数量的因素和仿冒应用的发展轨迹,每个视角都被进一步分解成了多个不同的研究问题。
在完成每个视角的解读后,本文均从数据中挑选一个具有代表性的案例进行分析,以案例进一步深化分析结果。

\fullref{chp:feedback} 提供了面向仿冒应用的排名欺诈检测方法。
针对现有检测方法的不足,本文先后提出了两个具有创新性的方法对排名欺诈行为进行排查。
结果显示,仿冒应用与排名欺诈行为作为移动黑灰产的两个环节确实存在关联。

\fullref{chp:future} 对本文工作进行总结,并对下一步工作进行展望。

\clearPaperPage

\chapter{背景知识介绍}
\label{chp:background}

随着人们对电子产品的焦点从桌面端慢慢转移到移动端,互联网黑灰产业也有向移动端趋向的迹象。
本章将按自底向上的顺序,分别介绍Android应用程序、应用市场和移动黑灰产相关的背景知识,以期让读者对移动黑灰产的整体生态有大致的了解。

\section{Android应用程序的构建与签名}

Android应用是移动黑灰产的终端环节之一,了解Android应用的构建和签名过程能帮助我们了解黑灰产从业者可以在哪一步植入牟利手段。

\subsection{Android应用程序构建}

与大部分软件一样,开发者在发布自己的App之前,也先需要把代码编译打包成Android操作系统使用的一种应用程序包格式文件,即APK(Android application package)。
每个APK文件都会包含该款App的一系列基本信息,包括App的应用名、包名(Package name)、安全证书等。
其中,包名是Android系统识别App的依据,每款App在不同的版本可以有不同的应用名,但其包名必须是一致的。
\autoref{fig:Android-Build-Process}展现了APK文件的构建流程。
一般来说,一个Android App的构建流程会分为以下四步,整个构建流程由Android SDK中的Android插件和Gradle构建工具管理。

\begin{figure}[htbp]
	\centering
	\includegraphics[width=0.6\textwidth]{./Figures/edwin-build-process-CHN.png}
	\caption{Android App构建流程}
	\label{fig:Android-Build-Process}
\end{figure}

首先,开发者需要编写App对应的源代码,然后连同一些源代码中使用到的依赖项一起输入到编译器中生成DEX文件。
源代码可以由Java语言或者Kotlin语言编写,而DEX文件则是一种可执行文件,可以运行于Dalvik虚拟机上。
Dalvik虚拟机则是Android系统的核心组成部分之一,用于运行被编译为DEX文件的程序。
此外,编译器还会将其他未被编译的资源文件转换为编译后的资源。
黑灰色产业从业者可以在这步就将恶意代码植入到应用中。

然后,SDK中的APK打包器会将DEX文件和已经编译好的资源文件一起打包。
APK文件的本质是压缩文件,其中包含了被编译的代码文件、App需要用到的资源文件(比如字符串、图片等资源)、assets资源、App的安全证书和Manifest配置文件,所以APK打包器的任务是将这些所有文件都压缩进一个APK文件里面。
不过,在这个步骤,APK打包器还未将所有文件压缩。
因为在压缩之前,还需要进行下一步的签名。

在第三步,APK打包器会使用密钥库文件对上一步中提及的资源文件和代码文件进行数字签名。
这个步骤是用作校验APK文件是否被篡改、保证APK文件完整性的一个重要步骤,但也是黑灰色产业从业者可以利用的一个环节。
下一小节会有相关机制的更多介绍。

最后,APK打包器会使用zipalign工具对应用进行优化,以减少App在设备上运行时所占用的内存。
这步结束之后,整个构建流程也随之结束。
开发者会获得一个编译好、签名完毕并且经过优化的APK压缩文件,然后就可以将这个APK文件安装到Android设备上运行使用。

目前,随着技术发展,互联网上还出现了各式各样的App生成器~\cite{anjian, iApp},用户无需了解复杂编程知识,只需简单操作即可实现App的开发。
一方面,这种简易的开发步骤有助于黑灰产从业者快速生产恶意应用;
另一方面,App开发框架的提供者本身也可能对生产出的App嵌入恶意代码。
业界相关报告~\cite{anquanke_framework}表明,黑灰产从业者对App生成器的滥用已经带来了严重的安全隐患。

\subsection{Android应用签名机制}
\label{sec:signature}

开发者在使用Android SDK构建App时,其中十分重要的一步是对App进行数字签名。
实际上,Android的数字签名和安全证书机制基于RSA公共密钥系统,是Android安全机制中不可或缺的一个部分。
本章的余下内容将会对Android App的签名机制进行简单分析。

Android App的签名机制是用作校验APK文件是否被篡改、保证APK文件完整性的一个重要机制,所有的应用都必须要在经过签名才能安装进Android系统中。
在签名时,SDK会使用一种密钥库文件,如果开发者还没有这个文件的话,SDK会自动生成一个。
密钥库中包含了开发者的各种信息,包括一对公钥和私钥。
私钥用于数字签名,不可向外公布;公钥则是可以向外公布的一组密钥,用于数字前面的验证。
App中的签名也是系统用来识别开发者的重要依据,因为同一个密钥库文件会产生一致的签名,系统能根据签名中的公钥验证应用识别开发者。

签名的过程大致如下:
在前文流程的第二步结束后,编译器会输出DEX文件和编译好的资源文件,这时,SDK会对每个文件都扫描一次,然后对每个文件提取一次数字摘要,再把每个文件的文件名和其对应的数字摘要保存在一个名为\textit{MANIFEST.MF}的文件中。
之后,SDK会再扫描一次刚才生成的\textit{MANIFEST.MF}文件,再次提取一次数字摘要,把这个摘要连同刚才文件中的所有内容存入另一个新文件\textit{CERT.SF}里。
第三步,再计算一次\textit{CERT.SF}的数字摘要,然后用密钥库中的私钥对这个摘要进行加密。
加密后的结果就是数字签名。
最后,SDK将签名、公钥、计算数字摘要的哈希算法等信息写入\textit{CERT.RSA}文件中,再将这整个过程中生成的四个文件放进\textit{META-INF}文件夹,用APK打包器打包起来。
至此流程结束。

而Android系统验证签名的方式,则是先通过\textit{CERT.RSA}中的公钥验证签名是否无误,再根据文件中提供的哈希算法计算APK包中所有文件的数字摘要:先从\textit{CERT.SF}开始,然后是\textit{MANIFEST.MF},然后是APK中的其他所有文件...
一旦其中出现不相符的结果,就会导致验证失败。
在安装App的过程中,验证签名失败会使得系统终止App的安装。

换句话说,在一个APK被打包签名完毕之后,如果需要更改其中的内容,就只能在更改后将APK重新打包签名一次,即使是一个bit的修改也会破坏原有的签名。
这也是系统可以用数字签名识别开发者的原因:签名一致的App,最后一定都是由同一个开发者打包的。
所以,具有同样签名的App也可以在同一个Android设备上共享数据。
不过这超出了本文讨论的内容,故按下不表。

目前,签名的模式共有三代,其区别主要在于构建流程第三、第四步之间的一些操作上。
简单地说,越新的签名模式能越好地保障APK文件的完整性。
实际上,第一代签名模式V1具有较为致命的缺陷,所以Google官方也在呼吁开发者在编译时采用最新的签名模式。

要注意的是,签名机制只能保证APK文件在被篡改之后不能凭借原有的签名被安装进Android系统,但恶意开发者依然可以在篡改APK之后,用自己的密钥库对APK重新签名,构建出可安装的App。
这种App是盗版App的一种,被称为重打包App。

另外,虽然一个安全证书只能指向一名开发者,但一个开发者可以同时拥有多个安全证书。
开发者和安全证书之间具有一对多的映射关系。

\section{Android应用市场}
\label{sec:androidMkt}

由于每个人都可以开发、构建自己的Android App,从网上发布的App数不胜数。这种开放性为Android应用生态带来开放性的同时,也会引入包括安全隐患等一系列问题。
而Google提供的Google Play应用商店无疑为用户和开发者都提供了一个优良的解决方案。
官方对上架前的应用审核,为用户安全提供了一定保障;商店中每个应用底下由用户评论组成的社区也促成了用户和开发者之间的交流,用户反馈直接推动了开发者对应用的改良。

遗憾的是,由于种种原因,Google Play应用商店的服务并非对全球的所有地区和国家都开放。
Google从2008年开始退出中国大陆市场,因此Google的大部分服务,包括Google Play应用商店的下载服务在内,都不向中国大陆境内用户提供。
换句话说,国内的大部分普通用户并不能享受到Google Play应用商店的便利。

为此,国内有多家厂商都推出了自己开发的应用市场服务,如腾讯旗下的应用宝~\cite{Myapp}和百度旗下的百度应用市场~\cite{Baiduappstore},还有华为的应用市场、小米的小米应用市场~\cite{Xiaomiappstore}等,试图填补这一片市场空白。

\begin{figure}[h]
	\centering
	\includegraphics[width=0.8\textwidth]{./Figures/edwin-yyb.jpg}
	\caption{腾讯应用宝应用市场首页(从桌面端浏览)}
	\label{fig:mkt-yyb}
	\vspace{-5mm}
\end{figure}

不同于在Android系统发布早期就存在的Google Play应用商店,国内的第三方应用商店是后期出现的产物,一出现就面临着激烈的市场竞争。
一方面,在国内各类第三方应用市场方兴未艾之时,国内的Android开发者社群尚未成熟,应用市场还未有大量开发者进驻;
另一方面,在成立初期,为了抢占市场份额,各个应用商店都想方设法将商店内App的种类和数量最大化,以迎合市场用户各种各样的需求。
作为结果,各类第三方应用市场都在各个渠道搜集App,而非通过开发者上传的方式获得货架上的应用程序。
由于在早期各种监管渠道、审核机制尚未完善,各个市场在搜罗各类App的同时,难免会将大量的盗版、恶意应用也一并收录。
换句话说,早期的各类第三方市场为移动黑灰产的成长提供了良好的温床。

\section{移动黑灰产简介}
依托Android端应用进行牟利的移动端黑灰色产业链条多种多样,在此我们向读者介绍与本次研究较为相关的两种:恶意应用与排名欺诈。

\subsection{恶意应用}
恶意应用是移动端黑灰色产业中最重要的一环。
作为与用户接触的终端,恶意应用也是多种形式的黑灰色产业的负载触发点。

要了解恶意应用产业,可以从三个问题下手:
恶意应用是怎么安装到用户的设备里的?
恶意应用都会有什么恶意行为?
这些恶意行为是如何触发的?

1)\ \emph{恶意应用的安装} \quad
相关研究~\cite{Zhou2012DissectingAM}表明,恶意软件的安装途径主要可以分为三个分类:重打包、更新攻击和路过式下载(Drive-by Download)。

重打包的概念在\secref{sec:repackaging}已经有所介绍。
为了诱使用户下载重打包后的应用,黑灰产从业者往往会选择热门的应用进行重打包,再在监管力度不太严格的应用市场进行重新发布。
这意味着,在我们的研究主体——仿冒应用——中,会有一部分重打包应用。
但是,迄今并没有相关研究揭示仿冒应用中重打包应用的占比,我们将在后文进行研究。

更新攻击是指恶意应用开发者为了躲避应用市场的监管审查,有意在应用市场上上架不包含恶意代码的应用,然后在用户安装应用之后,提示用户升级,进而绕开应用市场,将带有恶意代码的``新''版本安装到用户设备上的行为。
\autoref{fig:update-attack}给出了一个更新攻击的样本示例,图片引自Zhou与Jiang于2012年发布的研究~\cite{Zhou2012DissectingAM}。

\begin{figure}[h]
	\centering
	\includegraphics[width=0.7\textwidth]{./Figures/edwin-update-attack}
	\caption{更新攻击样本示例}
	\label{fig:update-attack}
	\vspace{-5mm}
\end{figure}

比起上述的两种安装途径,路过式下载更加隐秘。
路过式下载指在用户不知晓的情况下下载和安装恶意软件,和更新攻击相似,路过式下载行为的携带者(也就是一个表面上没有恶意代码的应用)可以被正常地上传到应用商店中,供用户下载。
等用户安装了对应应用之后,就有可能因为各种事件触发路过式下载,将不想要的应用甚至其他恶意应用静默安装到设备上。
这些事件可以是误触了应用中的某个弹窗广告,也可以仅仅是打开了无线网络开关,路过式下载甚至可以发生在设备每次启动完毕的时候。

2)\ \emph{恶意行为分类} \quad
恶意应用包含的恶意行为同样具有多样性。
以对设备侵入性从高到低排序,已知恶意行为可以被分为几个分类:特权提升,远程控制,恶意扣费和信息收集。

特权提升指恶意应用利用系统级别漏洞,获取其原本不应有、甚至超越用户级别的权限。
一旦有了这些权限,恶意应用就能对用户设备中的数据进行任意篡改,获取到系统级别权限的恶意应用甚至可以对用户设备进行任意操控,危害用户的数据安全。
具有这类行为的代表应用有形形色色来源不明的Root软件。

远程控制则利用了远程服务器对用户设备进行操纵。
具有这类恶意行为的应用都是木马程序,他们往往会在代码中隐含着一个C\&C服务器地址(命令和控制服务器,Command and Control Server)。
应用启动之后,就会在后台与服务器联系,接收并执行来自服务器的命令。
前段时间操控用户设备的Android挖矿应用就可以归入具有这类行为的应用中。

恶意扣费行为通常与电信服务运营商相关的订阅服务和短信收发权限有关。
在这类行为被触发时,恶意应用会在后台利用发送短信的权限向运营商发送服务订阅短信,部分恶意应用甚至会拦截运营商的订阅确认短信,对其屏蔽或者删除,使用户在不知情的状况下消耗收集资费。
这类行为的代表应用有被重打包的小游戏或者盗版的视频播放器。

信息收集行为最为普遍。
进行信息收集的应用会利用获得的权限搜集用户设备中的私人信息,如通讯录信息、位置信息、设备的唯一识别码(IMEI)等,再返回给恶意开发者。
恶意开发者可以对这些信息进行倒卖获利,也可以利用这类信息对手机用户精确画像,对用户进行精准营销。

3)\ \emph{负载触发条件} \quad
借助Android系统中各组件之间灵活多变的沟通手段,恶意应用负载(即恶意行为)的触发方式也有不同的种类。

以路过式下载触发方式为例,前文提及的点击应用内弹窗广告导致路过式下载可以算是主动触发方式,因为这需要用户主动点击应用中的控件才能触发;
打开无线网络开关和设备启动完毕就导致的路过式下载则属于利用应用监听系统广播触发的被动触发方式,这类应用会在配置文件中注册监听器,接收来自系统的广播信息,一旦捕获到相关的广播信息,就会运行恶意代码。

恶意应用的负载触发次数越多,恶意开发者的获利可能就越大,但同时恶意行为暴露的可能性也会越大。
为了躲避查杀和被用户察觉,有些恶意应用还会选择牺牲一定的触发频率,挑选十分苛刻的条件触发负载。
Pandita等人在2013发表的研究WHYPER~\cite{pandita2013whyper}就寻找到了一个触发条件苛刻的应用。
该应用只会在半夜12点后,在设备收到的移动网络信号发生变化时才会执行负载。

\subsection{排名欺诈}
应用市场的评论系统营造了一个社区环境,搭建了用户和开发者之间交流的平台,其他用户也能从其他用户的评价决定自己是否也要下载某款应用。
能从其他用户的反馈中获得参考固然是好事情,不少开发者也的确从用户的评论内容中得到了启发。
但不幸的是,移动黑灰产从业者从评论区中也发掘出了商机。

一些黑灰产从业者以应用搜索优化/应用商店优化(App Search Optimization/App Store Optimization,简称ASO)的名义,为一些不良商家提供操纵应用榜单排名的服务(即排名欺诈, Ranking Fraud)。
他们会为客户的App提供大量虚构的五星好评,刷高这些App的评分与在应用商店中的排名,而不管该款应用质量如何。
这种行为在令一些本身质优的应用被埋没的同时,也让一些广受好评的应用陷入信任危机,破坏了正常的市场秩序。

作为移动黑灰产生态中的两个独立环节,排名欺诈与仿冒应用是否会有所关联?
为了解开这个疑问,我们在实证研究后追加了评论分析的部分。

\section{本章小结}
本章主要整理介绍了Android应用的构建流程、Android安全证书机制、应用市场与黑灰产业的关系和一些已知的移动黑灰产知识。
黑灰产从业者通过直接编写恶意代码或者篡改原版APK的方式生产恶意应用,而国内早期第三方市场的监管不完善助力了移动黑灰产业的成长。
如果以恶意应用为代表的黑灰产环节进一步利用了排名欺诈行为提升其影响力,那么受其侵害的用户群体将很有可能被进一步扩大。

那么属于移动黑灰产环节之一的仿冒应用又是怎样的呢?这类应用有没有利用排名欺诈服务?目前仍未有已知课题对这个方面进行具体研究。
因此,我们将在本文的后续部分开展针对仿冒应用的大规模分析,以求获得仿冒应用生态的相关知识。

\clearPaperPage

\chapter{研究概览与数据收集}
\label{chp:dataCollection}

如前文所述,仿冒应用的生态是移动黑灰色产业研究中缺失的一环。
为了补上这一研究空白,在没有前人研究提供数据或分析的基础上,从工业界直接获取真实世界的样本来完成实证研究是最为直观合适的方法。
因此,本研究分四个部分,对仿冒应用进行实证研究。
本章将先为读者提供四个部分的研究概览,然后再详述其中的数据收集部分。

% \subsection{Workflow of Our Study}
\section{研究概览}

\begin{figure}[htbp]
	\centering
	\includegraphics[width=\textwidth]{./Figures/edwin-overview}
	\caption{本文研究流程}
	\label{fig:Workflow}
	\vspace{-3mm}
\end{figure}

% Fig.~\ref{fig:Workflow} shows the workflow of our study, which can be divided into two main phases:
\autoref{fig:Workflow} 展示了本次实证研究的流程,其可以分为四个主要阶段:

1)\ \emph{数据收集} \quad
数据收集主要分为两个部分:正版应用信息的收集和仿冒应用的收集。
在正版应用信息收集的部分,我们选择了50个最热门的App作为目标App,然后手动收集了跟这些App有关的信息;
仿冒应用信息收集方面,我们得到了上海犇众信息技术有限公司的帮助,顺利收集到了大量的应用样本。
借助自动化分析框架\mytool,我们获得了大量仿冒应用的数据。

2)\ \emph{大规模数据挖掘} \quad
在这一步,我们对搜集到的仿冒样本数据进行分析,从三个视角完成了一次大规模数据挖掘。
三个视角分别是仿冒的基本应用特征、影响仿冒应用数量的因素和仿冒应用的发展轨迹,由浅入深,帮助读者理解仿冒应用的生态。

3)\ \emph{仿冒案例分析} \quad
之后,我们从收集的数据中选择了3个案例进行更深入的分析。
这三个案例在支持数据挖掘中的发现之外,揭示了更多仿冒应用开发者的行为特征。

4)\ \emph{市场反馈分析} \quad
在这个部分,我们从第三方应用市场中随机选取一部分应用,提取了用户对它们的所有历史评价,然后筛选出其中对仿冒应用的评价并分析,以期了解用户对这些应用持有的态度。
此基础上,我们还检测了仿冒应用与排名欺诈行为的关系,揭示移动黑灰产不同环节之间的关联。

\section{应用数据收集}
鉴于目前学术界中未有相关工作能提供仿冒应用的相关数据集,我们率先尝试着从工业界中系统地收集我们需要的仿冒应用数据。
然而仿冒应用是一个跟正版应用相对的概念,所以我们需要先定义正版应用,再根据正版应用的信息搜寻仿冒应用。

\subsection{收集流程简介}
1)\ \emph{正版应用收集} \quad
在定义正版应用方面,我们参考了数据平台易观千帆的月度App排行榜\footnote{\url{https://qianfan.analysys.cn/refine/view/rankApp/rankApp.html}},然后从中选出了其中的前50款热门App作为我们的目标App。
之后,我们手动地从这50款App的官方网站上下载了这些应用的最新版本,作为正版应用的参考版本。

2)\ \emph{仿冒应用收集} \quad
要获取足量的仿冒应用数据以组成数据集是一个十分具有挑战性的任务,难点如下:
\begin{itemize}
	\item 我们要从多个不同的应用市场中爬取App样本,每个应用市场都有不同的网页编码,不存在一个爬虫脚本对所有应用市场数据都通用的场景;
	\item 各个应用市场架上的App数量浩如繁星,我们需要有效地找到和目标App有关的所有样本,不重不漏;
	\item 对于大量数据,我们需要一个轻量级的解决方案快速判断获得的App样本究竟为正版应用又或者是仿冒应用。
\end{itemize}

为了应对第一个挑战,我们与工业界合作,利用犇众信息公司的Janus平台对各大应用商店进行样本爬取,从而获得大量应用样本。
事实上,如\autoref{fig:Janus-data}显示,Janus平台自2017年起就开始对各大应用商店的App进行样本收集,至今已收集到上千万个App样本。
除样本搜集外,Janus也提供按规则搜索功能,用户可以创建自己的规则过滤平台中的应用数据,以获取自己需要的App样本。

\begin{figure}[htbp]
	\centering
	\includegraphics[width=\textwidth]{./Figures/edwin-Janus-data.png}
	\caption{Janus平台上的数据规模时序图}
	\label{fig:Janus-data}
	\vspace{-5mm}
\end{figure}

为了应对第二个和第三个挑战,我们搭建了应用筛选框架\mytool。
利用一个基于广度优先搜索(Breadth-First Search,简称BFS)的算法,我们在\mytool 中实现了一个\componentB ,以应用的\emph{包名}和\emph{应用名}作为拓展数据项,从Janus平台中分步迭代搜索与目标应用相关的所有App样本,然后将相关样本下载到本地保存;
对下载到本地的应用,我们使用\mytool 的组件\componentA 提取我们需要的数据;
然后,基于前述收集到的正版应用信息,我们在\mytool 中构建出了一个\componentC ,将上一步中提取到的数据和正版应用作比对。

关于应用数据收集流程的详细信息,可以参考\fullref{chp:fakerevealer}的介绍。

% \noindent {\bf Collected Dataset.}
\subsection{应用数据概览}
% Here is a bird eye's view to the data we collected:
在这里,我们先给收集到的数据提供一个数据概览:

% we chose the top 50 popular apps from Analysys's ranking, within 11 app categories, as our target apps. Since apps may change their names over time,  we recorded 198 app names from the 50 apps to mine fake samples.
从易观千帆提供的数据榜单中,我们选择了50个最热门的App作为研究主体,这些App分属11个不同的应用类别。
由于App的应用名可能会在App更新迭代的时候随之变更,我们用近似BFS的策略,从50种App中一共记录了198个不同的应用名,来挖掘仿冒样本。
% Among the 50 apps, we failed to find any samples from the following three apps: \emph{OPPO AppStore}, \emph{Huawei AppStore}, and \emph{MI AppStore}, because they are developed by cellphone manufacturers and are not provided to other app markets.
在这50款App中,我们并不能在市面上找到以下三款App的任何样本:\emph{OPPO 应用商店},\emph{华为应用商店}和\emph{小米应用商店}。
因为这三款App都是由手机设备厂商开发和预装在对应品牌的手机中的,仅供这些品牌的用户使用,并不在其他应用市场上提供下载。
% This is also the reason why three apps are popular -- they are preinstalled into every single device produced by their manufacturers.
当然,这也是这三款App热度高的原因——这几款App都被预装到了对应手机品牌厂商的每一部Android设备中,而OPPO、华为和小米又是国内最大的几家手机厂商,这几款App自然也会有庞大的用户基数。
% Thus we finally obtained 47 target apps in total.
因此,我们最后的目标App只有47款。

% With the 47 target apps, we retrieve 138,106 distinct samples in total, 69,614 of which are official samples of our target apps, 52,638 samples lack registered certificates.
对这47款目标App,我们总共收集到了138,106个应用样本。
其中,69,614个应用样本持有官方开发者证书,52,638个应用样本并不具有官方证书。
还有一部分应用样本,是某些应用的分别发布在不同应用市场同一版本,在经过我们的去重筛选后被排除(共计15,854个)。

% For each sample, we retrieve 8 data items as metadata: \emph{Sample SHA1}, \emph{Certificate SHA1}, \emph{Package Name}, \emph{Package Size}, \emph{Version Number}, \emph{Retrieved Time}, and \emph{Source}.
对于每个样本,我们会收集8个数据项作为元数据:\emph{样本SHA1码},\emph{安全证书SHA1码},\emph{包名},\emph{样本大小},\emph{版本号},\emph{搜集时间}和\emph{APK包来源}。
% Among them, \emph{Sample SHA1} and \emph{Certificate SHA1} are the hash code for APK files and certificates under SHA1 algorithm respectively.
其中,\emph{样本SHA1码}是使用SHA1哈希算法对整个APK文件进行数据摘要之后获取到的编码串,每个样本都有独一无二的SHA1码;安全证书SHA1码则是对样本的安全证书采用SHA1算法提取数据摘要之后获取的编码串,用于识别不同的证书。
% \emph{Retrieve Time} tells when the sample was crawled from app store and \emph{Source} tells which store the sample is from.
而\emph{搜集时间}则是样本从应用市场被爬取到数据库的时间点,\emph{APK包来源}指示该APK包来源的应用市场。

\section{本章小结}

本章主要对本研究进行了概览,介绍了数据收集的流程和方法,然后对采集到的数据进行了简要描绘。
% Empirical study is then applied to these metadata, especially to those of fake apps, to gain us a more comprehensive understanding on fake apps' nature and characteristics, and the behaviors of fake app authors.
下一章,我们将详述数据收集工具\mytool 的设计和实现。

\clearPaperPage

% \section{Large-scale Empirical Study and Discoveries}
\chapter{大规模实证研究与发现}
\label{chp:discoveries}

With the large-scale dataset ready, we can further conduct a comprehensive empirical study to acquire the nature of fake apps as well as understanding their ecosystem.
在大规模数据库准备好之后,我们可以
To effectively measure different facades of fake apps, We define three perspectives, namely \emph{Fake Sample Characteristics}, \emph{Quantitative Study on Fake Samples}, and \emph{Developing Trend}.
Next, we'll describe each perspective in detail.

% \subsection{Fake Sample Characteristics}
\section{Fake Sample Characteristics}
To reveal the strategy the fake app authors are employing, or how they bypass app markets' security scheme, fake sample characteristics have to be understood.
As such, we conduct our measurement in terms of certificates and basic information like app names, package names and package sizes.

Certificate serves as the identifier for developers.
The nature of the certificate, namely, whether each fake app has a unique certificate, is likely to be essential to fake apps' evasive technique.
On the other hand, we believe repackaged apps, as a kind of \texttt{imposters}, are widespread in our dataset.
Measurement on basic information of fake apps, such as package names and package sizes, helps us determine how repackaged apps are distributed, since repackaging an app does not change any of its basic information (i.e. the app name, package name, version code, etc.) unless it's done intentionally.

To this end, we have some hypotheses as below:

\noindent{\bf Hypo 1.1:} Most of these fake samples have their corresponding unique certificates.
In other words, most fake certificates and fake samples have a one-to-one relation.

\noindent{\bf Hypo 1.2:} A large portion of fake samples have the same app names/package names/apk sizes as those in official samples.

To verify our hypotheses as well as to gain knowledge to fake sample individuals, we propose the following research questions in this subsection.

\noindent{\bf RQ 1.1}: What's the relationship between the number of fake samples and their certificates? That is, how many fake samples does one certificate usually link to?

\noindent{\bf RQ 1.2}: How do fake apps imitate official apps? That is, how similar are the names/package names/apk sizes of fake samples compared to those of official samples?

\begin{figure*}[htbp]
	\centering
    \subfloat[应用名\label{fig:appname}]{\includegraphics[width=0.33\textwidth]{./Figures/edwin-RQ1-2(a).png}}\hfill
    \subfloat[包名\label{fig:pkgname}]{\includegraphics[width=0.33\textwidth]{./Figures/edwin-RQ1-2(b).png}}\hfill
    \subfloat[样本大小\label{fig:size}]{\includegraphics[width=0.33\textwidth]{./Figures/edwin-RQ1-2(c).png}}\hfill
	\caption{对App各项属性的统计结果}
	\label{fig:Statistic_fake_and_official}
	\vspace{-5mm}
\end{figure*}

\noindent{\bf Answer to RQ 1.1.}
76\% of these fake certificates are linked to merely one or two fake samples, and the number of fake examples a certificate links to is various from 1 to 1,374.
We count the number of certificates which link to different sample number in table~\ref{table:certificate_number_statistic}.

\begin{table}
  \renewcommand{\arraystretch}{1}
  \footnotesize
  \centering
  \caption{Statistics on fake samples and their certificates}
  \vspace{1mm}
  \begin{tabular}{l c c c c c c c}
  \toprule
  {\bf \# of samples} & {\bf 1-5} & {\bf 6-10} & {\bf 11-50} & {\bf 51-100} & {\bf More than 100} \\
  \midrule
  {\bf \# of certificates} & 8252 & 525 & 531 & 71 & 80 \\
  \bottomrule
  \end{tabular}
  \label{table:certificate_number_statistic}
\end{table}

This discovery partly matches our assumption that most of these fake samples have their corresponding unique certificates.
We consider this as a strategy to bypass app markets' security scheme, as even if one fake sample is exposed, other fake samples developed by the same developer will not be implicated directly.
Nevertheless, when reviewing certificates linked with multiple fake samples, we find some very surprising findings that we will expound in Section~\ref{sec:casestudy}.

\noindent{\bf Answer to RQ 1.2.}
According to our statistical result, only 243 out of 52,638 samples (less than 0.5\%) use official package names, all the rest fake samples (more than 99.5\%) use their own package names.
In the rest 52,395 samples, 14,089 different package names were found.
But does this mean fake samples are all using package names that are totally different from the official ones? Could they be using package names that are similar to their official correspondences?

To figure out the similarity, we utilize \textit{edit distance}~\cite{levenshtein1966binary}, a distance definition widely applied in natural language processing (NLP):
{Given two strings $a$ and $b$, the edit distance $d(a, b)$ is the minimum-weight series of edit operations that transform $a$ into $b$.
In our case, edit operations refer to either to append, to delete or to change a character.}
For instance, the edit distance between string ``fake" and ``official" is 7, while between ``jingdong" and ``jindeng", this value becomes 2.
For every fake package name from a fake sample, we compute its edit distance to the official package name of its original.

Fig.~\ref{fig:Statistic_fake_and_official} is consist of three violin plots,\footnote{\url{https://en.wikipedia.org/wiki/Violin_plot/}} representing our statistics on app names, package names and package sizes, respectively.
In each ``violin'', the white dot represents the median, the thick bar in the middle represents the interquartile range while the thin bar represents 95\% confidence interval.

Fig.~\ref{fig:appname} shows the statistic information on app names of official samples, fake samples, and the edit distance between them.
Both the white dot in ``Official'' violin and the one in ``Fake'' violin are at a similar level near the value ``6'', which means the average length of app names of both official samples and fake samples are close to each other.
The overall distribution of these two data groups have similar bodies, signals that they are also similar as well.
What's more, the median value of edit distance is low (``2'' on $y$-axes), meaning that half of the fake apps get their names by modifying less than 3 characters from the corresponding official apps' names.
This is a proof indicating that most fake apps are using a similar name to an official name.
At the same time, we notice that some fake apps have pretty long names (there is one with a name of 146-character-long length).
Many of those outliers are samples uploaded by fake authors, maybe for testing purpose to explore the vetting mechanism. The other purpose is to associate users' search keywords as far as possible.

Fig.~\ref{fig:pkgname} shows the result on package names.
Like the plots in Fig.~\ref{fig:appname}, the difference between the average length of package names of official apps and the average length of package names of fake apps is still tiny (they are of value ``23'' and ``20'', respectively).
Nonetheless, the median of edit distance between them is explicitly higher (``16'' on $y$-axes), which means it takes averagely 16 times modification to turn a fake package name to an official package name and vice versa.
Thus, we infer that fake apps tend to use self-defined package names.

Fig.~\ref{fig:size} reports package size information.
To better represent the trend, we eliminated some outliers: samples that are larger than 150MB (851 in 69,614 official samples (about 1\%) and 447 in 52,638 fake samples (less than 1\%), most of which are from 游戏 category).
The figure shows that the median number of fake samples' size is around 5MB, while half of the official apps have a size greater than 18MB, meaning that fake apps are more likely to be
(1) developed by their owners but not originated from repackaging official apps,
(2) malicious apps, for malicious apps are usually in small sizes.

In short, Fig.~\ref{fig:Statistic_fake_and_official} tells that fake apps
(1) prefer to use a similar (or even same) name to an official app's name, but they have their own package names and
(2) are usually of a small size.

To a large extent, we owe the first point to the incompleteness of the information the app store displays on apps.
In most app stores, when users browse an app's detail page, they can only see the app's name, description, user comments and ratings which are positive for leading users to download that app.
However, technical information rarely appears.
In some app markets, users don't even know how large an apk file is.

\vspace{1mm}
\noindent\fbox{
	\parbox{0.95\linewidth}{
		\textbf{Remark 1}: Most certificates link with only a number of fake apps, which is highly possible to be a fake developers' evasive strategy.
			Moreover, we observe that fake apps do tend to use official app names or names alike.
			Nonetheless, fake apps and official apps are not resemble in terms of package names or apk sizes, disclosing that repackaged apps are not mainstream in fake apps.

	}
}

% \subsection{Quantitative Study on Fake Samples}
\section{Quantitative Study on Fake Samples}
It is valid to assume that fake app developers are driven by profits, hence there is a likelihood that the number of fake app is correlated to their source market, popularity and categories.
In addition, the update frequency can be taken in as a factor, too.
Accordingly, we hypothesize the following factors may influence the number of fake samples of an app:

\noindent{\bf Hypo 2.1:} {The rate of fake samples is related to the number of apps a market contains.}

\noindent{\bf Hypo 2.2:} The number of fake apps are closely related to how popular an app is.

\noindent{\bf Hypo 2.3:} Update frequency effects the number of fake samples.

\noindent{\bf Hypo 2.4:} Category is a factor influencing the fake sample number.

Correspondingly, we define our research questions as follows:

\noindent{\bf RQ 2.1}: Where are these fake samples mainly from?

\noindent{\bf RQ 2.2}: Does the popularity of an app affect the number of its fake samples?

\noindent{\bf RQ 2.3}: Does an app's update frequency influence its fake sample's number?

\noindent{\bf RQ 2.4}: Is the number of fake samples related to the app's category?

\begin{ThreePartTable}
\centering
\renewcommand{\arraystretch}{1.05}
\footnotesize
\setlength{\belowcaptionskip}{-5pt}
\vspace{1mm}
% \rowcolors{2}{gray!15}{white}
\begin{longtable}{l l c c c c c c}
\caption{目标App与其相关统计}\label{table:data-statistics}\\
\toprule
{\bf 应用名} & {\bf 类别} & \begin{tabular}[c]{@{}c@{}}{\bf 月度热} \\ {\bf 度指数} \end{tabular} & \begin{tabular}[c]{@{}c@{}}{\bf 更新频率} \\ {\bf (天/版本)} \end{tabular} & {\bf 样本总数} & \begin{tabular}[c]{@{}c@{}}{\bf 仿冒} \\ {\bf 样本数} \end{tabular} & {\bf 仿冒率} & \begin{tabular}[c]{@{}c@{}}{\bf 平均仿} \\ {\bf 冒延迟} \end{tabular} \\
\midrule
{\bf 微信}\tnote{*} & {\bf 社交网络} & {\bf 91.2K} & {\bf 6.4} & {\bf 9248} & {\bf 6447} & {\bf 69.7\%} & {\bf 12.1} \\
\rowcolor{gray!15} {\bf QQ}\tnote{*} & {\bf 社交网络} & {\bf 54.6K} & {\bf 10.7} & {\bf 11167} & {\bf 3780} & {\bf 33.8\%} & {\bf 9.2} \\
爱奇艺 & 视频 & 53.5K & 6.4 & 7586 & 3481 & 45.9\% & 9.3 \\
\rowcolor{gray!15} 支付宝 & 生活 & 48.1K & 10.2 & 983 & 231 & 23.5\% & 10.1 \\
{\bf 淘宝}\tnote{*} & {\bf 移动购物} & {\bf 47.5K} & {\bf 7.0} & {\bf 6003} & {\bf 3010} & {\bf 50.1\%} & {\bf 8.1} \\
\rowcolor{gray!15} 腾讯视频 & 视频 & 47.3K & 6.3 & 1429 & 68 & 4.8\% & 10.7 \\
优酷 & 视频 & 40.9K & 7.3 & 2058 & 262 & 12.7\% & 6.7 \\
{\bf 新浪微博}\tnote{*} & {\bf 社交网络} & {\bf 39.2K} & {\bf 5.3} & {\bf 5947} & {\bf 2715} & {\bf 45.7\%} & {\bf 5.7} \\
\rowcolor{gray!15} WiFi万能钥匙 & 系统工具 & 36.4K & 3.1 & 4808 & 2999 & 62.4\% & 3.0 \\
搜狗输入法 & 系统工具 & 33.3K & 11.0 & 898 & 40 & 4.5\% & 21.8 \\
\rowcolor{gray!15} 百度 & 资讯 & 32.4K & 11.1 & 15651 & 3514 & 22.5\% & 12.8 \\
腾讯新闻 & 资讯 & 28.7K & 8.5 & 1051 & 11 & 1.0\% & 8.9 \\
\rowcolor{gray!15} QQ浏览器 & 资讯 & 27.8K & 5.6 & 1369 & 43 & 3.1\% & 11.6 \\
今日头条 & 资讯 & 27.4K & 4.4 & 3538 & 179 & 5.1\% & 5.6 \\
\rowcolor{gray!15} 应用宝 & 应用市场 & 27K & 11.4 & 2419 & 266 & 11.0\% & 11.6 \\
快手 & 视频 & 24.4K & 3.2 & 8273 & 4270 & 51.6\% & 3.5 \\
\rowcolor{gray!15} 腾讯手机管家 & 系统工具 & 24.2K & 8.7 & 2463 & 1340 & 54.4\% & 8.7 \\
高德地图 & 生活 & 24K & 6.5 & 1225 & 51 & 4.2\% & 13.1 \\
\rowcolor{gray!15} 酷狗音乐 & 音乐 & 23K & 8.6 & 1313 & 122 & 9.3\% & 12.2 \\
QQ音乐 & 音乐 & 21.7K & 9.4 & 1132 & 65 & 5.7\% & 14.6 \\
\rowcolor{gray!15} 百度地图 & 生活 & 21.3K & 8.8 & 2609 & 1438 & 55.1\% & 15.3 \\
抖音 & 视频 & 19.4K & 11.1 & 317 & 12 & 3.8\% & 8.3 \\
\rowcolor{gray!15} {\bf 京东}\tnote{*} & {\bf 移动购物} & {\bf 18.5K} & {\bf 10.9} & {\bf 5000} & {\bf 2377} & {\bf 47.5\%} & {\bf 12.3} \\
UC浏览器r & 资讯 & 16.7K & 7.4 & 4232 & 1624 & 38.4\% & 7.0 \\
\rowcolor{gray!15} 360手机卫士 & 系统工具 & 15.4K & 12.4 & 3670 & 1423 & 38.8\% & 19.1 \\
全民K歌 & 音乐 & 14.7K & 21.1 & 618 & 215 & 34.8\% & 17.3 \\
\rowcolor{gray!15} 美团 & 生活 & 13K & 8.0 & 4752 & 1415 & 29.8\% & 6.9 \\
{\bf 拼多多}\tnote{*} & {\bf 移动购物} & {\bf 12.9K} & {\bf 6.6} & {\bf 2327} & {\bf 551} & {\bf 23.7\%} & {\bf 7.8} \\
\rowcolor{gray!15} {\bf 王者荣耀}\tnote{*} & {\bf 游戏} & {\bf 12.5K} & {\bf 15.5} & {\bf 2350} & {\bf 1319} & {\bf 56.1\%} & {\bf 12.3} \\
美图秀秀 & 摄影录像 & 12.4K & 5.4 & 1705 & 784 & 46.0\% & 5.8 \\
\rowcolor{gray!15} 火山小视频 & 视频 & 12.2K & 11.9 & 410 & 16 & 3.9\% & 9.6 \\
墨迹天气 & 生活 & 12K & 4.2 & 10081 & 7093 & 70.4\% & 4.7 \\
\rowcolor{gray!15} 滴滴出行 & 生活 & 11.8K & 8.6 & 943 & 117 & 12.4\% & 7.0 \\
华为应用市场 & 应用市场 & 11.8K & N/A & 0 & 0 & 0.0\% & N/A\\
\rowcolor{gray!15} {\bf 开心消消乐}\tnote{*} & {\bf 游戏} & {\bf 11.2K} & {\bf 19.7} & {\bf 2406} & {\bf 1738} & {\bf 72.2\%} & {\bf 20.6} \\
酷我音乐盒 & 音乐 & 11K & 2.9 & 3778 & 69 & 1.8\% & 4.2 \\
\rowcolor{gray!15} 西瓜视频 & 视频 & 11K & 11.5 & 866 & 100 & 11.5\% & 8.8 \\
OPPO应用商店 & 应用市场 & 10.8K & N/A & 0 & 0 & 0.0\% & N/A\\
\rowcolor{gray!15} 猎豹清理大师 & 系统工具 & 9.9K & 10.3 & 1803 & 388 & 21.5\% & 13.5 \\
360清理大师 & 系统工具 & 9.6K & 17.3 & 327 & 8 & 2.4\% & 8.5 \\
\rowcolor{gray!15} 360手机助手 & 应用市场 & 9.2K & 7.6 & 1616 & 137 & 8.5\% & 8.4 \\
WiFi管家 & 系统工具 & 8.8K & 19.5 & 1636 & 658 & 40.2\% & 15.7 \\
\rowcolor{gray!15} 讯飞输入法 & 系统工具 & 8.6K & 6.0 & 1451 & 8 & 0.6\% & 10.1 \\
百度手机助手 & 应用市场 & 8.2K & 11.4 & 3849 & 437 & 11.4\% & 14.5 \\
\rowcolor{gray!15} 小米应用市场 & 应用市场 & 7.8K & N/A & 0 & 0 & 0.0\% & N/A\\
{\bf WPS Office}\tnote{*} & {\bf 商务办公} & {\bf 7.4K} & {\bf 6.0} & {\bf 1152} & {\bf 69} & {\bf 6.0\%} & {\bf 7.8} \\
\rowcolor{gray!15} 美颜相机 & 摄影录像 & 7.1K & 5.3 & 1600 & 691 & 43.2\% & 6.3 \\
网易云音乐 & 音乐 & 7K & 10.5 & 616 & 6 & 1.0\% & 12.2 \\
\rowcolor{gray!15} 网易新闻 & 资讯 & 6.7K & 7.0 & 1441 & 93 & 6.5\% & 5.0 \\
{\bf QQ邮箱}\tnote{*} & {\bf 商务办公} & {\bf 6.6K} & {\bf 16.4} & {\bf 520} & {\bf 11} & {\bf 2.1\%} & {\bf 10.4} \\
\bottomrule
\end{longtable}
\begin{tablenotes}
  \footnotesize
  \item[*] Detailed descriptions are given in {\bf Answer to RQ 2.4}
\end{tablenotes}
\vspace{-3mm}
\end{ThreePartTable}


\noindent{\bf Answer to RQ 2.1.} Fig.~\ref{fig:Sample_source} shows the samples' origin.
From the left subplot, \texttt{Baidu App Store} not only provides the largest sample number among all 31 different app sources, but is also the source where most fake samples are from.
Fake sample rates are displayed on the right subplot.
Although both \texttt{Baidu App Store}~\cite{Baiduappstore} and \texttt{Hiapk}~\cite{Hiapk} hold a fake sample rate of about 40\%, the number of fake samples from \texttt{Baidu App Store} exceeds \texttt{Hiapk} to a great extent due to its dominant total sample number.
%\edwin{mark.}
Although no connection between the number of fake samples and market can be found from our data, we notice that the relationship between apps and markets may affect the fake rate.
This is well supported by the low fake rate of \texttt{Myapp}~\cite{Myapp} -- the app market provided by \texttt{Tencent}, which is also the 12 out of 50 developers in our target apps.
%\edwin{mark ends here.}

\begin{figure*}[htbp]
	\centering
  \setlength{\belowcaptionskip}{-10pt}
	\includegraphics[width=\textwidth]{./Figures/edwin-Number_of_samples_collected_markets_3.png}
	\caption{Number of samples collected from different markets}
	\label{fig:Sample_source}
\end{figure*}

\noindent{\bf Answer to RQ 2.2.}
Intuitively, the more popular an app is, the more possible it would get shammed, for fake developers would mislead users to download their apps to gain profits.

Note that each app has different amount of samples (including official samples and fake samples), processing our measurement directly based on the number of fake samples is incorrect.
To counteract this bias, each fake count should be regularize into a \textit{fake sample rate}, the rate of fake samples in all collected samples of an app.

Next, we employ a metric called \textit{Pearson product-moment correlation coefficient (PPMCC)} to reveal relativeness between an app's fake sample number and its popularity, which uses the regularized fake sample rates and monthly activeness indicators (MAI) obtained from Analysys~\cite{yiguanqianfan}.
This value ranges from -1 to 1, the closer the PPMCC value is to 0, the weaker correlation between the two factors is indicated.
Surprisingly, according to our data, the value of PPMCC between this two factors is 0.246, revealing that the fake sample number and an app's popularity only hold their relativeness on a weak level, which does not match our expectation.

\noindent{\bf Answer to RQ 2.3.}
We assume the update frequency is related to the number of fake samples of an app, for updates can usually help keep a software from being attack.
The higher the update frequency is, the safer an app is supposed to be.

To estimate the average update frequency of our target apps, the time when an app's official sample was crawled and when its latest official samples were crawled is marked.
The difference between them is then divided by the number of that app's existing version to obtain an update frequency, with unit day/version.

The result PPMCC value of 0.084 shows that the connection between an app's update frequency and its fake sample rate barely exists.
We attribute this result to two reasons:
(1) The high update frequency (10 days/version on average for apps in our dataset) indicates app developers may not fix security issues in per update, weakening the function of update frequency as a security indicator.
(2) A large portion of fake samples in our dataset are not derived from repackaging. To this end, fake developers can produce fakes regardless of how well the official apps are protected.

\noindent{\bf Answer to RQ 2.4.}
Some categories are potentially more profitable than others.
A report from the app marketing company LIFTOFF~\cite{LIFTOFF_report} forecasts gaming to be the next most billable area.

Our 50 target apps are divided into 11 categories according to their functionalities, Table~\ref{table:data-statistics} shows these categories and their corresponding fake sample rate.
In the same category, the difference between apps on fake rate lies in an acceptable range.
Without doubt, entertainment related categories like \texttt{游戏} and \texttt{Social Network} attract more fake samples.
Another field, \texttt{Online shopping}, has also gained special love from fake developers because online shopping is rapidly developing in China.
Relatively, \texttt{商务办公} is not that attractive to fake developers, the average fake sample rate of this filed is only 4.05\%.
Apps in these four categories are marked in bold in the table.

The result matches the observation in our daily life, people always tend to use mobile devices for entertainment instead of business purpose.
It's pretty interesting to discover that the number of fake samples in a way reflects how people use their phone in their daily lives.

\vspace{1mm}
\noindent\fbox{
	\parbox{0.95\linewidth}{
		\textbf{Remark 2}: As revealed by statistics, the number of samples returned from an app store does not imply a fake rate.
		Additionally, the relationship between apps and market itself influences the number of fake samples from that market.
		To our surprise, an app's update frequency is not tightly correlated with its fake rate.
		We owe this to the fact that apps are updated too frequently and that repackaged samples are of minority in our dataset.
		We further observe that ``category'' as a factor has greater influence on the number of fake samples of an app than ``popularity'' and ``update frequency''.

	}
}

\begin{figure}
	\centering
	\includegraphics[width=\textwidth]{./Figures/edwin-Number_of_samples_collected_per_quarter_3.png}
	\caption{Numer of fake samples collected per quarter}
	\label{fig:Number_per_quarter}
\end{figure}

% \subsection{Developing Trend}
\section{Developing Trend}
In order to figure out fake apps' characteristics or behavior patterns over time, we propose the following research questions:

\noindent{\bf RQ 3.1} After a new version of an official app is published, how long do fake developers take to publish a new fake sample? In other words, how soon will these copycats appear?

\noindent{\bf RQ 3.2} How long can a fake app's certificate survive?

\noindent{\bf RQ 3.3} Is there a changing pattern of fake samples over time?


\noindent{\bf Answer to RQ 3.1.}
We compute this latency and show its distribution in Fig.~\ref{fig:Fake_latency_overall_distribution}.

Due to various reasons, it is hardly possible to retrieve the complete updating timeline for every single official app in our study, yet we approximately reproduce them with our data.
Firstly, we categorized all the official samples by their origins, and further categorized samples in each origin by version number.
After that, for each app and each version the samples are sorted by the date they were crawled, so by extracting the crawled date of the first sample in each version, we can obtain the earliest date a version is released.
Lastly, by combining and sorting the release dates of different versions according to different apps, we can reproduce the updating timelines of our target apps.

To find out the release latency of a fake app, all the dates on the timeline of the corresponding official app are compared in order to find out the smallest negative difference which we define as the release latency.
Fig.~\ref{fig:Fake_latency_overall_distribution} shows that most fake samples are published with the latency shorter than 20 days.
According to our statistics, 60\% of fake samples show up in 6 days after a new version of the official app is published.
This reveals a truth that fake developers are swift in action.


\noindent{\bf Answer to RQ 3.2.} Fig.~\ref{fig:Fake_certificate_survival_distribution} shows the distribution of the time a fake certificate can survive in markets.
In the left density distribution subplot, $x$-axes is the latency and $y$-axes shows the probability density of data at corresponding $x$ value.
%\edwin{mark.}
The total area under the curve is 1, and the area under two $y$ values $y1$ and $y2$ is the probability of their corresponding value $x1$ and $x2$ account for in data.
For example, in Fig.~\ref{fig:Fake_certificate_survival_distribution}, the area beneath curve between 0 to 200 on $x$-axes is close to 0.8, which means nearly 80\% of certificates only survive for no more than 200 days.

To judge how long a fake certificate can survive is similar to how we calculate the update frequency of an app, the first time and the last time a fake sample from the same certificate gets crawled are marked.
The time when a sample was crawled from a market might be different from the time when it is available in the market, but our crawler downloads new samples from different markets by days and we also use days as the unit in our measurement, so we can approximately regard this two values as the same one.

As shown in Fig.~\ref{fig:Fake_certificate_survival_distribution}, the distribution of fake certificate survival time shows that almost all the fake certificates live a short life, which means most fake certificates only show up in a short period of time.
This can be explained by a scheme that most markets have.
Once an app is found malicious or illegal, the market would stop that specific developer from uploading more samples by refusing to receive samples with the same certificate.
There are also a number of certificates which can survive for a long time.
According to the figure, some fake certificates even traverse the whole study interval.
We will conduct a case study on this phenomenon in Section ~\ref{sec:casestudy}.

\begin{figure}
	\centering
	\includegraphics[width=\textwidth]{./Figures/edwin-Fake_certificate_survival_distribution2.png}
	\caption{Fake certificate survival time distribution}
	\label{fig:Fake_certificate_survival_distribution}
\end{figure}

\noindent{\bf Answer to RQ 3.3.} Fig.~\ref{fig:Number_per_quarter} shows the number of fake samples collected per quarter since the fourth quarter of 2015.
Although a large number of new fake samples get released in every quarter, the figure shows a tendency that the total number of fake apps on markets is gradually decreasing by years.
Note that our statistics only focus on fake samples, consequently this phenomenon does not indicate the underground industry is turning down.
Instead, we suppose this is possibly caused by the reform of fake apps.

On one hand, as stricter review schemes and stronger protection systems are applied on app stores, it's inevitable that fake apps in this study, become harder and harder to get on the shelf.
On the other hand, the new generation of malicious software, such as ransomware~\cite{ransomware} is impacting the underground industry.
Compare to fake apps, the new malicious apps are not only hard to defend (due to the innovative or even state-of-the-art techniques they utilize) but also extremely profitable.
Wannacry, a ransomware which was first spotted in the 2nd quarter of 2017, conquered tens of thousands of devices in a couple of weeks, which directly pulled up Bitcoin's price like a rocket~\cite{wannacry_bitcoin_news}.
Afterward, in the first quarter of 2018, a burst of cryptomining malware on phones emerged~\cite{comodo_report}.
This may be the reason why the number of fake samples suffers two suddenly drops in the second quarter of 2017 and the first quarter of 2018, respectively.


\begin{figure}
	\centering
	\includegraphics[width=\textwidth]{./Figures/edwin-Fake_latency_overall_distribution2.png}
	\caption{Fake latency overall distribution}
	\label{fig:Fake_latency_overall_distribution}
\end{figure}

\vspace{1mm}
\noindent\fbox{
	\parbox{0.95\linewidth}{
		\textbf{Remark 3}: Fake apps can be produced in a relatively short time, and the dropping number of fake samples by years suggests that they are mired in recession.
		Besides, only a few fake certificates survive for a long time, confirming that markets' protection schemes do work to some extent.
	}
}

\clearPaperPage

% \chapter{Case Study and Discussion}
\chapter{案例分析与讨论}
\label{chp:casestudy}
% In this section, we present some samples in our dataset, not only to firm our findings but also to provide more valuable insights.
在本章中,我们会呈现数据集中的一些样本,不仅仅是为了确认我们在上一章中提及的各种发现,同时也是为了提供一些更有价值的见解,让读者对我们的数据集、对仿冒应用有更加深刻的认知。

% \noindent{\bf Case study 1.} \emph{Fake certificate with multiple malicious imitators and imposters}
\section{案例 1. 与大量仿冒应用相关联的仿冒应用安全证书}
\label{sec:case1}

% We manually review the samples signed by the certificate with SHA1 ``\emph{61ed377e85d386a8dfee6b864bd85b0bfaa5af81}", the certificate with the most number of fake samples among our fake certificate set (i.e., 1,374 fake samples).
在\secref{sec:fakeCharacteristics}中,我们提到有部分仿冒应用安全证书关联着多个仿冒样本。
其中,一个SHA1码为``\emph{61ed377e85d386a8dfee6b864bd85b0bfaa5af81}''的安全证书是所有证书中关联仿冒样本最多的,足足有1,374个样本持有这个仿冒证书。
% On top of that, this certificate is also one of the certificates survive the longest time (nearly 3 years) and is still active.
不仅如此,这个安全证书还是其中一个在应用市场中存活时间最长的仿冒应用安全证书。
它的出现横穿了我们的整个研究时间过程(接近三年),而且在我们的数据收集流程结束前依然呈现活跃状态。

% Originally, we presume this certificate to belong to a benign app which passes the verification of analysts in Pwnzen, since the number of samples it links to even exceeds the number of official samples of some apps.
最初,我们假定这个证书属于某个通过犇众分析团队验证的良性应用,因为它关联的样本数量实在是太大了,这个数量甚至超过了我们某些目标App的样本总量。
% The truth is, however, after manually review, we found all the 1,374 samples linked with this certificate are typical fake samples, in form of either imitators or imposters, covering 79\% (37 out of 47) of our target apps.
然而在人工审核之后,我们却发现了意料之外的结果。
这个证书关联的所有1,374个样本都是典型的仿冒样本,其中既有只与原版应用稍微近似的\texttt{模仿应用},也有外观上完全模仿原版应用的\texttt{高仿应用},覆盖了我们能找到的47个目标App中的37个(79\%)。
% Some of its samples can even be organized in version order, which means the developer does track official apps to update its fake versions as maintenance.
而这些证书关联的一些样本,甚至有自己的版本顺序,这表明有的开发者真的会追踪官方App的各个版本来更新、甚至维护自己的仿冒版本。

\begin{table}[htbp]
	\renewcommand{\arraystretch}{1}
	\small
	\centering
	\caption{由``61ed377e85d386a8dfee6b864bd85b0bfaa5af81"签署的部分样本}
	\vspace{1mm}
	\rowcolors{2}{gray!15}{white}
	\begin{tabular}{l l c c c c c c}
		\toprule
		{\bf Name} & {\bf Package Name} & {\bf Size} \\
		\midrule
		QQ Talk  & net.in1.smart.qq & 465.8 KB \\
		QQ  & com.h & 8.2 MB \\
		爱微信  & com.lovewechat & 368.4 KB \\
		微信  & com.tencen1.mm & 22.1 MB \\
		UC Mini  & com.uc.browser.en & 2.1 MB \\
		UC 浏览器  & com.UCMobile.microsoft & 21.3 MB \\
		Clean Master  & com.blueflash.kingscleanmaster & 972.0 KB \\
		WiFi万能钥匙  & com.snda.wifilocating & 5.9 MB \\
		\bottomrule
	\end{tabular}
	\label{table:certificate_case_study}
\end{table}

% We display some of the samples singed by this certificate in Table~\ref{table:certificate_case_study}, they are all reported to be malicious (i.e., Ad-ware, spyware or Trojan) on \textsc{Virustotal}~\cite{virustotal}, a famous online antivirus engine.
我们在\autoref{table:certificate_case_study}中展现了由这个安全证书签署的部分样本。
我们将这些样本上传到知名在线反病毒引擎\textsc{Virustotal}~\cite{virustotal}上,结果显示,与该证书关联的所有样本都是恶意样本(广告软件、间谍软件或木马软件等)。
% So far the samples related to our target apps have already been showing up in 20 markets including the leading ones like \texttt{\small Myapp} and \texttt{\small Qihoo 360 Market}.
到目前为止,由这个证书签署的仿冒样本已经在我们的20个应用来源(即应用市场)中出现,包括\texttt{应用宝}和\texttt{360应用市场}等主流应用市场。
% What's more, \texttt{\small Baidu App Store} keeps receiving apps with this certificate from 2015 to recently -- its latest ``product'' was put on shelf on September 15$^{th}$, 2018.
除此之外,\texttt{百度手机助手}从2015年起就开始接受由这个安全证书签署的应用,直到我们的数据收集阶段结束前——数据显示,这个安全证书在2018年9月15日还在\texttt{百度手机助手}上架了一款应用。

% To this end, we can draw the following conclusions:
在这里,我们可以得到以下两个结论:
\begin{itemize}
	% (1) Even the leading app markets (and the top developers) are unqualified in detecting malicious apps.
	\item 就算是领先的应用市场(和顶尖的开发者)在检测恶意应用方面也不能做到尽善尽美,而现有的检测方法也有所不足,未能及时地找出可疑的开发者;
	% (2) Existing app markets lack information exchange on defending attacks from underground industry.
	\item 从这个证书在多个市场都存在的现象,我们可以推导,现有的应用市场缺乏有效的信息交换机制。如果各个应用市场能建立一个互通信息的平台,分享可疑开发者/恶意开发者信息,那么将可以杜绝一部分恶意开发者在各个应用市场上到处流窜的现象。
\end{itemize}

% \noindent{\bf Case study 2.} \emph{Fake samples in different gaming apps}
\section{案例 2. 游戏类别下的仿冒应用}

% Gaming apps in our target app list (i.e. \texttt{\small ArenaofValor} and \texttt{\small HappyElements}) attract a number of fake samples.
正如\autoref{table:data-statistics}中数据所示,\texttt{游戏}类应用(\texttt{王者荣耀}和\texttt{开心消消乐})吸引了大批的仿冒应用样本。
% To figure out what do these samples look like, we randomly downloaded some of the fake samples of the 2 gaming apps (7 samples for each) and installed them on our testing device.
出于性能考虑,我们的数据中只提取了APK包的基本信息,并没有对收集到的每个APK文件进行详细的分析。
因此,为了弄清楚这些仿冒应用究竟是怎么样的,我们随机从这两款游戏App的仿冒样本中选择了一些样本(每款应用选择7个仿冒样本),然后将这些样本安装到了我们的实验设备上。

% Fig.~\ref{fig:screenshot_all} shows how these samples look like on a real Android phone, official apps are marked with green frames.
\autoref{fig:screenshot_all}展示了这些样本在真实的Android设备上安装之后的实际外观。
官方渠道下载的正版App在图片中由绿色边框标记出。
% Apparently, fake samples have either a similar name or a similar icon to official ones.
明显地,我们能看到,与官方应用相比,仿冒应用要么就有一个和官方应用名十分相似的应用名,要么就在图标上和官方相近甚至相同。

\begin{figure}[htbp]
	\centering
    \subfloat[两种游戏App与其仿冒\label{fig:screenshot_all}]{\includegraphics[width=0.3\textwidth]{./Figures/edwin-screenshot1.jpg}}\hfill
    \subfloat[正版\textit{\small 开心消消乐}\label{fig:screenshot_official}]{\includegraphics[width=0.3\textwidth]{./Figures/edwin-screenshot2.jpg}}\hfill
    \subfloat[仿冒版\textit{\small 开心消消乐}\label{fig:screenshot_fake}]{\includegraphics[width=0.3\textwidth]{./Figures/edwin-screenshot3.jpg}}\hfill
    \label{游戏类App及其仿冒样本}
\end{figure}

% We even ran these apps on our device.
我们甚至在测试设备上实际运行了上述安装的14个仿冒样本,然后拿他们和原版的应用对比。
我们使用的测试设备是高配版的小米5手机,搭载的CPU为最高主频2.15GHz的骁龙820处理器,3GB内存,64GB机身存储,安装的Android系统版本为Android 6.0(Android Marshmallow,API 23)。

% Screenshots were captured when we ran one of the fake samples (see Fig.~\ref{fig:screenshot_fake}) and the official sample (Fig.~\ref{fig:screenshot_official}).
\autoref{fig:screenshot_official}和\autoref{fig:screenshot_fake}分别是在我们在测试设备上运行官方版本的\texttt{开心消消乐}和其中一个仿冒版的\texttt{开心消消乐}时的系统截屏。
不难看出,两款应用的外观是十分相像的。
我们在测试时发现,两款游戏内部的玩法、实际操作逻辑也一模一样。
如果不是事先知道了哪一款应用是来自官方渠道下载的正版,连我们都没有办法判别两个应用的真伪,更不必说是从应用市场搜索结果中找到这些结果的普通用户了。

% As a result, we found that 4 fake samples of \texttt{\small HappyElements} are actually games that are similar to the official one (one is a repackaged app with high confidence), 2 are raiders on the game and the last one crashed when it was launched.
而这并不是唯一的案例。作为结果,我们发现7款\texttt{开心消消乐}的仿冒样本中,有4款是与官方样本十分相似的游戏(其中一个十分可能是经过重打包技术处理的应用),2款声称自己是``系统攻略'',还有1款在运行时闪退,无法在我们的设备上实际运行。
% 3 out of the 4 fake games pop up alert windows in the game to require users for In-App purchase, which is very possible to cause unwilling cost.
在4款仿冒游戏中,3款都在游戏中不时自动弹出游戏内购窗口,要求玩家购买道具,十分可能导致玩家不想要的花费。
% All 7 samples are reported to be malicious on \textsc{Virustotal}~\cite{virustotal}.
而所有7个仿冒样本都在\textsc{Virustotal}~\cite{virustotal}中被报告为恶意应用。

% Fake samples on \texttt{\small ArenaofValor}, in contrast, barely have functionalities like the official one.
相比之下,\texttt{王者荣耀}的仿冒样本内容就与官方应用大相径庭了。
% 3 of those samples are wallpaper setters and the rest 4 are simply puzzle games.
在7款被安装到测试设备的仿冒样本中,有3款是壁纸浏览器,里面包含了几张游戏内人物的插画,可以在应用内将这些插画设置成系统的桌面壁纸;
而余下四个是简单的拼图游戏,里面同样包含了王者荣耀游戏人物的插画,应用内容就是简单地把被打乱的插画拼图恢复原状。
% Virustotal reports 6 out of the 7 samples as malware, the last one is claimed as potentially unwanted program (PUP).
Virustotal的结果显示,7款仿冒样本中,有6款是恶意软件,涵盖了木马病毒、广告软件等类型,而余下的一款则被报告为潜在有害程序(Potentially Unwanted Program,简称PUP)。
PUP通常在用户不知情或者不愿意的情况下,通过静默安装或者捆绑安装的形式被安装在系统中。
尽管这种软件不一定包含恶意代码,但其动机十分可疑。

% We determine it is the difficulty to imitate the official app's functionality that brings about this phenomenon.
我们认为,这种现象是由模仿正版应用功能的难易程度带来的。
% The core implementation of multiplayer online battle arena (MOBA) games like \texttt{\small ArenaofValor} is much more complicated than that in \texttt{\small HappyElements}.
只从技术角度看,像\texttt{王者荣耀}这样的多人在线战斗竞技场(Multiplayer Online Battle Arena,简称MOBA)游戏核心难度明显要比\texttt{开心消消乐}这样的益智类游戏要高得多。
一款MOBA游戏除了要解决支持运行运行的物理引擎之外,还要实现聊天系统、在线匹配、负载均衡等业务,更加不必说背后的人物设计技能平衡等更深入的话题了;而一款益智类三三消游戏的核心逻辑就只在于元素三连的判定和随机新出现的元素,再加上道具系统就差不多可以包装成一个完整的游戏推出。

% What it costs to develop a complex game like \texttt{\small ArenaofValor} is exorbitant for a fake developer.
因此,就算不考虑后续的维护问题,要开发一款像\texttt{王者荣耀}这样的复杂游戏,对仿冒应用开发者来说明显是成本过高的。
但由于这款游戏本身具有超高的热度,可能带来巨大的收益,所以仿冒应用开发者会为了蹭上热度而开发外观相似、内容完全不符的仿冒样本。
相比之下,\texttt{开心消消乐}由于开发难度相对较小,所以仿冒应用开发者会愿意开发一个内容相似的应用,再通过内购陷阱等手段收取效益。
这两款App透露出了仿冒应用开发者在仿冒方面两个截然不同的思路。

% Therefore, we can infer another reason why apps in \texttt{\small productivity} category gain a low fake sample rate:
我们在\secref{sec:quantitativeStudy}中观察到的\texttt{商务办公}类别有较低仿冒率也可能是类似原因导致的结果。
% Unlike games, on one hand, productivity tools are less likely to have peripheral products (like wallpaper setter mentioned above);
一方面,\texttt{商务办公}类的工具核心逻辑比较复杂,对仿冒开发者来说并不是一个有利可图的最佳选项;
% On the other hand, the inevitably laborious developing procedure also prevents the tools themselves from being shammed.
另外,这类应用也不像\texttt{游戏}一样会衍生出周边产品(比如\texttt{王者荣耀}的游戏人物就会有不少插画),仿冒应用开发者也没办法从这方面入手蹭热度。
结合两个原因,\texttt{商务办公}类的应用自然就不会引起仿冒应用开发者的太多兴趣了。

% \noindent {\bf Case study 3.} \emph{Suspicious samples with official certificate}
\section{案例 3. 持有官方安全证书的可疑样本}

% Case study 1 gives us a perfect example on counterintuitive data in our data set.
\secref{sec:case1}的案例1提供了我们数据集中一个明显反直觉的示例。
% In order to find out whether or not resemble cases exist in our official samples, we manually reviewed them and noticed a weird entry when sorting out the sample log, a sample claims itself to be ``cracked" in its app name.
因此,我们不禁会好奇数据集中是否会有其他违反直觉的数据存在。
为了解开这个疑问,我们决定先从搜集到的正版样本入手。
我们手动浏览了所有69,614个持有官方证书的样本,其中有一个样本的应用名十分可疑——该样本声称自己是一个``破解版''的应用。
% Furthermore, we checked (1) if strange word (e.g., ``cracked") appears in our official samples' names, (2) whether or not an official app is signed by an official certificate from another developer, and (3) if one official sample has a suspicious package name.
于是,我们再次针对所有正版样本进行了以下几项筛选:
\begin{enumerate}
	\item 使用``破解''、``免费''等关键字搜索所有带正版证书的样本,筛选出带有可疑应用名的样本;
	\item 筛选所持安全证书与原开发者不一致的样本;
	\item 筛选出包名和同款App的多数样本不一致的样本。
\end{enumerate}

% Eventually we acquired 17 suspicious official samples, listed in table~\ref{table:suspicious_samples} are samples in each of these three kinds.
最终,我们获得了17个由正版开发者安全证书签署的可疑样本,其中三个样本的信息如\autoref{table:suspicious_samples}中所示,分别代表上述三项筛选得到的结果。
第一个名为\texttt{爱奇艺}的样本虽然由一个官方安全证书签名,但该安全证书和其他爱奇艺样本的却不一致。对比之后,我们发现该证书来自360手机助手,但360和爱奇艺并没有合作关系,因此这是个可疑的样本;
而第二个样本(\texttt{360手机助手})的可疑之处在于样本包名。多数\texttt{360手机助手}的包名为\emph{com.qihoo.appstore},也有少部分官方包名为\emph{com.qihoo.secstore},前者为\texttt{360手机助手}在国内第三方应用市场发行的应用包名,后者为Google Play官方应用市场上上架的包名。然而,其中一个使用了其官方安全证书签署的样本的包名却是\emph{com.kuyou.sdbgj.baidu},十分奇怪;
第三个样本则是在应用名中包含了``破解''字样。然而,正常的正版应用根本不会有这样的命名方式,所以我们也认为这是一个可疑样本。

% \textsc{Virustotal} reports that only 2 of the 17 samples are benign, 2 are PUP and the other 13 samples are all malicious.
\textsc{Virustotal} 的检查结果显示,17个可疑样本中,只有2个是良性应用,2个是PUP,余下13个样本都被判定具有恶意行为。

\begin{table*}[htbp]
	\renewcommand{\arraystretch}{1}
	\small
	\centering
  \setlength{\belowcaptionskip}{-10pt}
	\caption{持有官方安全证书的可疑样本}
	\begin{tabular}{l l c c c c c c}
		\toprule
		{\bf 样本应用名} & {\bf 样本SHA1码} & {\bf 可疑之处} \\
		\midrule
		% 爱奇艺 & b86c55a509e8293b24138b166e9ff410f39e84b5 & 可疑证书(360手机助手) \\
		爱奇艺 & b86c55a509e8293b24138b166e9ff410f39e84b5 & 可疑证书\\
		% 360手机助手 & 2bb43c53b86d204d0040a8af6cb2a09cf9e93bb7 & 可疑包名(com.kuyou.sdbgj.baidu) \\
		\rowcolor{gray!15} 360手机助手 & 2bb43c53b86d204d0040a8af6cb2a09cf9e93bb7 & 可疑包名\\
		% Youku XL 破解版 & b55b7ef189d649aeb03443c5d1ab57c9031d624e & 可疑应用名(``破解版") \\
		Youku XL 破解版 & b55b7ef189d649aeb03443c5d1ab57c9031d624e & 可疑应用名 \\
		\bottomrule
	\end{tabular}
	\label{table:suspicious_samples}
\end{table*}

% Despite the possibility that these certificates were somehow leaked to the underground industry, it is more likely that some attackers penetrated the protection scheme.
鉴于这17个样本都是持有官方安全证书签名的,我们起初不禁怀疑是否有应用厂家不慎泄露了自己的安全密钥库,从而导致了这些样本的出现。
然而,如果真的是因为厂家泄露密钥库,一来很有可能会导致恶意开发者使用官方安全证书大量生产恶意应用,二来对应厂家也会出于安全考虑马上更换新的包名和安全证书。
我们的数据并不支持以上猜想带来的两点结果,所以我们不认为这是由于安全证书泄露导致了这些可疑样本的产生。

除去这个可能性,我们认为更有可能的原因是某些仿冒应用开发者掌握了穿透/绕过Android系统签名机制的技术,从而产生了这些样本。

% As far back as December 2017, Google had confirmed and revealed a backdoor on V1 signature scheme (CVE-2017-13156)~\cite{android_security_bulletin}, by which hackers can inject any content into an apk at will without modifying its certificate information.
时间回溯到2017年12月,Google确认并公布了V1版本应用签名机制的一个后门(CVE-2017-13156)~\cite{android_security_bulletin}。
通过这个后门,黑客可以在不修改APK包安全证书信息的情况下,向APK包里注入任意内容。
% An alternative solution, V2 signature scheme, has been launched at least one year before that.
而早在这个漏洞被公布的至少一年之前,Google就已经发布了作为V1版签名机制的替代解决方案,也就是V2版应用签名机制。
% In order to confirm if these apps are using the risky V1 scheme, we used a tool, \textsc{apksigner}, provided by Google to verify which signature schemes these samples are using.
这看起来十分有可能是导致这些可以样本产生的原因,某些恶意开发者利用了V1版本签名机制的漏洞,修改了APK包的基本信息。
为了确认这些样本是否采用了具有风险的V1版应用签名机制,我们使用了\textsc{apksigner}来检测这些样本使用的签名机制版本。
\textsc{apksigner}是Google官方提供的一个命令行工具,它被集成在Android SDK中,既是APK包编译打包过程中为APK包进行数字签名的工具,也可以用来验证APK包使用的签名机制版本,又或者是验证APK的签名是否有效。

% It ends up that all 17 samples are using V1 signature scheme.
\textsc{apksigner}的结果显示,所有17个样本都只使用了V1版本的应用签名机制。
% With actually knowing that V1 is no longer safe, developers still refuse to embrace the safer scheme, which is really disappointing.
在了解到V1版本签名机制已经不再安全的情况下,仍有部分开发者由于各种原因没有接受更新也更安全的签名方案,这个结果有点令人失望。

\section{本章小结}
本章从数据集中选出了一些较有代表性又或者反直觉的数据样本,为读者提供了3个不同的案例分析,在为我们在上一章的发现提供有力支持之外,也揭示了更多仿冒应用开发者的行为特征。
回看三个案例,我们不难发现,仿冒应用开发者的确会抓住一切可能的机会,利用包括签名机制漏洞、市场审查机制缺陷在内的各种办法制作出仿冒甚至是恶意应用。
同时,本章的三个案例也说明了无论是开发人员还是应用市场,都应该为保护Android的软件安全上投入更大精力,从而更好地防范来自移动黑产的各种攻击。

\clearPaperPage

\chapter{仿冒应用的评论分析}
\label{chp:feedback}

应用市场中的用户反馈区是Android应用生态的重要部分,也是移动黑灰产从业者的深耕之地。
热心用户会在评论区提出反馈,黑灰产从业者则会利用排名欺诈的手段牟取利益。
为了进一步对仿冒应用的生态有所了解,作者有必要搜集和分析仿冒应用的评论。
对于与仿冒应用相关的评论,有以下待解答的问题:
有多少用户会对这些应用进行评论?
用户对这些应用的使用感受如何?
仿冒应用是否会跟排名欺诈有所联系?

为了回答这些问题,作者从\texttt{360手机助手}应用商店中爬取了部分应用和它们的所有历史评论,然后对这些评论进行了处理和分析。

\section{仿冒评论数据收集}
由于Janus平台上并不提供评论数据,所以作者重新在应用市场上收集评论数据。
在\texttt{360手机助手}应用商店中,作者随机挑选了856个应用,爬取了这些应用的APK包和它们的所有历史评论。
之后,作者将上一次数据收集时保存的正版应用的信息重新导入\mytool,并从这一批应用中筛选出了对应仿冒应用。

每条评论都会附带一个评价分数,某款App在市场上的平均评价就是所有评论评分的均值。
对于每条评论信息,作者能收集到的数据项是:\texttt{应用包名}、\texttt{用户ID}、\texttt{评论内容}、\texttt{评分}和\texttt{评论日期}共五项。

\subsection{评论数据概览}
在作者收集评论的所有856个应用中,有6款应用与先前仿冒应用数据中的包名对应。
要注意的是,由于本研究的仿冒应用列表为针对50个热度最高的目标App整理而成,而收集评论的应用是在整个市场范围内随机挑选的,所以这里的仿冒应用占总体应用比例较小。但这不意味着整个市场中就只有这几款应用是仿冒应用。
另外,由于这856个应用是随机挑选的,作者认为这批数据具有一定的代表性,可以应以反映整个市场的评论分布情况。
本次研究一共爬取到了267,397名用户的365,461条评论,其中6款仿冒应用的所有历史评论共计3,591条,来自2,946名用户。

\subsection{基础分析}

% 画图用的python代码
% font_size = 20
% sns.set(style="white")
% fig = plt.figure()
% plt.xticks(fontsize=20)
%
% ax1 = fig.add_subplot(111)
% bar = sns.barplot(x="Package name", y="Comment cnt", data=df, ax=ax1, palette="Blues_r")
% ax1.tick_params(axis='y', labelcolor=“tab:blue”)
% plt.yticks(fontsize=font_size)
% ax1.set_ylabel("# Comments", color="tab:blue", size=font_size)
%
% ax2 = ax1.twinx()
% line = sns.lineplot(x="Package name", y="Rating", data=df, ax=ax2, c="r", linewidth=3)
% ax2.tick_params(axis='y', labelcolor=“tab:red”)
% ax2.set_ylabel("Rating", color="tab:red", size=font_size)
% plt.yticks(fontsize=font_size)
% # 折线图y轴范围
% plt.setp(line, yticks=[4, 4.2, 4.4, 4.6, 4.8])
% # 旋转x轴标签
% for axis in [ax1, ax2]:
%     for tick in axis.get_xticklabels():
%         tick.set_rotation(10)
%
% # 清除上方右方边框
% sns.despine()
% plt.show()

\begin{figure}[htbp]
	\centering
	\includegraphics[width=\textwidth]{./Figures/edwin-cmt-ratings-dist-3.png}
    \caption{各仿冒应用在\texttt{360手机助手}商店中的评论数量与评分分布}
    \label{fig:cmt_dist_fake}
\end{figure}

\autoref{fig:cmt_dist_fake}显示了6款应用收到的评论数量和评分分布。
蓝色的柱状图表示评论数量,红色的折线图表示各评论汇总后的平均评分(以5分为满分算),$x$轴分别代表不同样本的包名。
按评论数从多到少的顺序看,\emph{com.jhwl.fjxa}收到了2223条评论,平均评价为4.98分;
\emph{com.arsenal.FunWeather}收到了626条评论,平均评价也是4.98分;
\emph{com.sscwap}收到了467条评论,平均评价4.76分;
\emph{edu.qust.weather}收到了223条评论,平均评价4.49;
\emph{com.intimateweather.weather}和\emph{com.cleaner.main}分别只收到了49条和3条评论,而他们的平均评价则分别为4.63分和5分。
乍眼一看,上述应用的评分都十分高。
以下是一些热门应用的评分和这些仿冒应用的评分的对比:
在同一市场下,移动购物类应用\texttt{淘宝}的评分为4.55分,近年十分受欢迎的短视频应用\texttt{抖音}平均评价是4.5分,游戏类应用\texttt{开心消消乐}的评分是3.65分,而\texttt{微信}的平均评价更是只有3.45分。
上述应用毫无疑问都是十分优质的App,庞大的用户基数带来的大量真实评价会使得平均评价较为稳定,不会因为在短时间内收到少量好评或者差评就产生较大的评价波动。
在\texttt{淘宝}等应用作为基准的情况下,6款仿冒应用的评分之高不禁令作者想到恶意刷评的相关研究。
但本研究不能仅凭几个应用的平均评价对比就咬定仿冒应用存在刷好评的行为,所以本文会应该先分析数据集的整体分布再作进一步比较。

在不考虑上述提到的恶意刷评的情况下,一款应用的使用人数越多,就越有可能收到来自用户的评价。
所以可以在一定程度上,从一款应用的评论数目估计其用户数量的多少。
\autoref{fig:cmt_dist_total}中的两个小提琴图显示了所有856个应用收到评论的数量和总体评级分布,这有助于让读者了解市场中应用的热度分布情况和用户的评价倾向。

% Python 作图代码
% font_size = 20
% sns.set(style="whitegrid")
% fig = plt.figure()
%
% ax1 = fig.add_subplot(121)
% bar = sns.violinplot(x=df2["Cmt cnt"], ax=ax1)
% ax1.set_xlabel("Package #Comment Distribution", size=20)
% plt.xticks(size=20)
%
% ax2 = fig.add_subplot(122)
% line = sns.violinplot(x=df2["Pkg Rating"], ax=ax2)
% ax2.set_xlabel("Package Rating Distribution", size=20)
% plt.xticks(size=20)
%
% # 清除上方右方边框
% sns.despine()
% plt.show()

\begin{figure}[htbp]
	\centering
	\includegraphics[width=\textwidth]{./Figures/edwin-360-comment-dist.png}
    \caption{所有856个应用在商店中的评论数量与评分分布}
    \label{fig:cmt_dist_total}
\end{figure}

左边的小提琴图表示每个应用收到的评论总数分布,其四分位数分别为31, 124和375.75。
这说明,在收集到的856个应用中,25\%的应用收到小于或者等于31条评论,50\%的应用收到小于或等于124条评论。
如果某款应用收到的评论数大于375条,那这款应用的评论总数就能排在前25\%了。
另外,数据显示,仅有5\%的应用收到了超过2,109条评论。
结合仿冒样本的数据,\emph{com.jhwl.fjxa}、\emph{com.arsenal.FunWeather}和\emph{com.sscwap}的评论数量都排在了前25\%,\emph{com.jhwl.fjxa}更是能排在前5\%的位置。
从数据上看,上面三款App相当受欢迎。

右边的小提琴图则表示各个应用收到的平均评价的分布情况,其四分位数分别为3.84,4.37和4.69,约5\%的应用平均评价为5分满分。
这个分布说明这个应用市场上的用户十分倾向于给出高分评价,至少有过半数的评论都是满分好评。
回到仿冒样本的数据,其中有两款评论少于100条,很可能存在较大的个体偏差,在此先忽略不计。
余下四款仿冒应用中也有三款的平均评价排进了前25\%,恰好也是评论数较多的\emph{com.jhwl.fjxa}、\emph{com.arsenal.FunWeather}和\emph{com.sscwap}。

结合两个维度的分布结果,\emph{com.jhwl.fjxa}、\emph{com.arsenal.FunWeather}、\emph{com.sscwap}三款App不仅用户众多,而且还好评如潮。
再回望\texttt{淘宝}、\texttt{微信}等应用的评分,两者之间似乎有了矛盾。
一款应用的用户越多,真实的评论数目越大,该款应用的评分就会趋向客观,直到收敛到一个可以反映应用质量的真实水平。
\emph{com.jhwl.fjxa}、\emph{com.arsenal.FunWeather}、\emph{com.sscwap}三款应用的用户量固然不能和\texttt{微信}、\texttt{淘宝}相比,但成百上千的评论数也暗示着一个不小的用户群体。
在用户群体有一定规模的情况下,还能保持名列前茅的平均评价,究竟是这三款仿冒应用的确受到了用户的热捧,还是另有原因?

\section{仿冒应用与排名欺诈关联验证}

\subsection{排名欺诈检测初探}

带着上述的疑问,作者决定用自动化的方法试图寻找这几款应用是否有排名欺诈的可能。

\texttt{FRAUDAR}~\cite{hooi2016fraudar}是Bryan Hooi在2016年推出的一个算法,可以使用二分图挖掘的方法找出可能的虚假好评,并输出最可能涉及排名欺诈的应用和用户。
算法基于的假设是,普通用户的行为(在本文中即为对App评论的行为)是大致随机的,而用于进行排名欺诈的用户群的行为却会有比较明显的指向性(即针对购买了排名欺诈服务的App发送好评),而且为了将平均评分拉高,就意味着需要发送大量好评。
如果将用户和App分别看做两种不同节点,每条好评看作是两种节点之间的边,那么在这张用户-评论-应用图中,排名欺诈用户和对应的App之间就会有特别紧密的联系。
如果能把这个联系特别紧密的子图找到,就有可能从中找到真实的排名欺诈用户和应用。

由于此处的排名欺诈指用大量虚假好评刷高应用的平均评价,所以在寻找排名欺诈用户和对应应用时,应该只采用满分好评作为数据。
因此,本研究从所有评论中筛选出了满分好评,其总数为381,507条,由229,100名用户给出,分布于848个应用中,占评论总数的87.15\%。
作者将\texttt{FRAUDAR}应用在了本研究的好评数据集上,找到了115名可疑用户和13个可疑应用。
经过比对后,作者发现,13个可疑应用并不包含仿冒样本,而仿冒应用的所有评论条目中也没有源于那115名可疑用户的评论。
但\emph{com.jhwl.fjxa}的评价表现依然令人生疑,因此,作者决定使用其他办法检测这几款仿冒应用的评论中是否存在排名欺诈的可能性。
基于本工作中现有的数据项,进行检测的角度可以分为两个,一个是评论用户可信度,另一个则是评论内容相似度。

\subsection{基于评论用户可信度的排名欺诈排查}

Mohammad-Ali~\cite{abbasi2013measuring}在2013年提出了一个个体行为相似度的计算方法。
在该算法中,用户行为相似度由\autoref{equ:usr_cre1}计算,如果相似度超过了某个阈值$T_1$,就可以将两名用户聚入同一类。
\autoref{equ:usr_cre1}中的$B(u_i, t)$指用户$u_i$在时间节点$t$的行为(在此处可以理解为对某一个应用给好评),而$\sigma(B(u_i, t), B(u_j, t))$则是一个用来计算用户$u_i$和$u_j$在时间为$t$时行为相似度的方程,Mohammad在文中选用的是\autoref{equ:usr_cre2}所示的Jaccard相似系数,所以本文在这里也选用了同样的Jaccard系数计算。

\begin{equation}
Sim(u_i, u_j) = \frac{1}{t_n - t_0}\sum_{t=t_0}^{t_n}\sigma(B(u_i, t), B(u_j, t))
\label{equ:usr_cre1}
\end{equation}
\begin{equation}
Jaccard(set_i, set_j) = \frac{|set_i \cap set_j|}{|set_i \cup set_j|}
\label{equ:usr_cre2}
\end{equation}
\vspace{0.5mm}

在这种计算方式下,那些仅给过一次好评的用户将会很容易成为噪声数据,对研究的结果产生影响,所以作者先剔除掉了这部分数据。
仅给过一次好评的用户共186,775人,占所有给出好评用户的81.53\%。

在将行为相似的用户聚类之后,可以通过计算某一应用评论中疑似排名欺诈评论的占比、或是评论该App的可疑用户占所有可疑用户的占比来排查可能购买了排名欺诈服务的App。

\begin{Def}
	应用的用户可信度权重

	假设所有市场用户的集合为$G_{all}$,已知疑似排名欺诈用户群体$G_r$,用户以$u$表示,由任意用户$u_i$发布的评论$cmt_j$表示为$u_i \rightarrow cmt_j$,市场中的某一应用$app_k$的评论列表为$CL_{app_k}$。
	则该应用$app_k$的用户可信度权重$W_{app_k}$可由\autoref{equ:usr_cre3}中的二元组表示:
\end{Def}

\begin{equation}
	W_{app_k} = (w_{app_k}^0, ~w_{app_k}^1)
	\label{equ:usr_cre3}
\end{equation}
\begin{equation}
	w_{app_k}^0 ~ = ~ \frac{|\{u_i~|~u_i \in G_r, cmt_j \rightarrow u_i, cmt_j \in CL_{app_k}\}|}{|G_r|}
	\label{equ:usr_cre4}
\end{equation}
\begin{equation}
	w_{app_k}^1 ~ = ~ \frac{|\{cmt_j~|~cmt_j \in CL_{app_k}, cmt_j \rightarrow u_i, u_i \in G_r\}|}{|CL_{app_k}|}
	\label{equ:usr_cre5}
\end{equation}
\vspace{0.5mm}

在计算完权重之后,可以分别按其中的两个子权重对应用进行排名,筛选出可能购买了排名欺诈服务的App。

作者分别将$T_1$设置为0.4,0.6和0.8,尝试对可疑用户进行聚类,结果分别将42,325名给出好评次数大于1的用户分到了10,024,14,520,15,493个聚类中。
可对这些聚类进行简单分析如下:
绝大多数聚类中都只有一名用户,即使是在相似度阈值只有0.4的情况下,也只有约7\%的聚类中包含3个或以上的用户(阈值为0.6和0.8时,该比例均为6\%)。
但是,当挑出包含用户数目大于10的聚类($T_1$为0.4/0.6/0.8时,这些聚类的占比分别为0.97\%/0.70\%/0.57\%)时,作者却发现这些聚类分别包含了29,168/23,742/22,218个用户,他们所发布的好评共计分别有98,226/80,492/73,422条。
因此作者推定,有一部分用户的行为模式相当近似且可疑,本文将会把这部分用户组成的群体看作是疑似的排名欺诈用户群体($G_r$)。

接下来,作者分别计算了不同$T_1$下,三款仿冒应用在\autoref{equ:usr_cre4}和\autoref{equ:usr_cre5}中的两个权重,即三款应用中的好评用户占可疑用户的比例、以及其好评占所有可疑用户发布的好评的比例。
对于本章研究的三个仿冒应用,其两个子权重的结果分别展示在\autoref{table:usr-cred-res-1}和\autoref{table:usr-cred-res-2}中。

\begin{table}[htbp]
	\renewcommand{\arraystretch}{1}
	\small
	\centering
	\caption{各应用用户可信度权重及对应排名(一)}
	\vspace{1mm}
	\begin{tabular}{lcccccc}
		\toprule
		包名 & $w^0$($T_1$=0.4) & 排名 & $w^0$($T_1$=0.6) & 排名 & $w^0$($T_1$=0.8) & 排名 \\
		\midrule
		com.arsenal.FunWeather & 0.97 & 2 & 0.9 & 6 & 0.82 & 9 \\
		\rowcolor{gray!15} com.jhwl.fjxa & 0.85 & 15 & 0.84 & 12 & 0.8 & 12 \\
		com.sscwap & 0.02 & 303 & 5$\times10^{-3}$ & 346 & 5$\times10^{-3}$ & 318 \\
		\bottomrule
	\end{tabular}
	\label{table:usr-cred-res-1}
\end{table}

\begin{table}[htbp]
	\renewcommand{\arraystretch}{1}
	\small
	\centering
	\caption{各应用用户可信度权重及对应排名(二)}
	\vspace{1mm}
	\begin{tabular}{lcccccc}
		\toprule
		包名 & $w^1$($T_1$=0.4) & 排名 & $w^1$($T_1$=0.6) & 排名 & $w^1$($T_1$=0.8) & 排名 \\
		\midrule
		com.jhwl.fjxa & 0.05 & 11 & 0.06 & 10 & 0.07 & 11 \\
		\rowcolor{gray!15} com.arsenal.FunWeather & 0.02 & 45 & 0.02 & 39 & 0.02 & 39 \\
		com.sscwap & 2$\times10^{-4}$ & 261 & 8$\times10^{-5}$ & 279 & 9$\times10^{-5}$ & 251 \\
		\bottomrule
	\end{tabular}
	\label{table:usr-cred-res-2}
\end{table}

表中结果显示,无论是用哪种权重对应用可疑度进行排名,\emph{com.arsenal.FunWeather}和\emph{com.jhwl.fjxa}在总计的848个应用中都排在相当靠前的位置,所以这两个应用都相当可疑,十分可能具有排名欺诈行为。
另一方面,\emph{com.sscwap}的排名相对靠后,具有排名欺诈行为的可能性较小。

本组实验使用了服务器承担运算任务,实验服务器搭载了两颗Intel的至强E5-2367 V4版8核CPU,内存为252GB。
在$T_1$分别设置为0.4/0.6/0.8时,三组基于用户可信度实验用的python代码分别需要运行7,086/6,935/6,801分钟才得出结果。

\subsection{基于评论内容相似度的排名欺诈排查}
与前面的用户可信度计算相比,利用评论内容相似度排查排名欺诈的方法要相对简单一些,计算量也明显较小。
根据作者的经验,在排名欺诈相关的评论通常具有很高的相似性,甚至一模一样,导致那些购买排名欺诈服务的应用中有很多相似甚至相同的评论。
所以,可以通过计算应用内相似评论的比率以筛选可能购买了排名欺诈服务的应用。

\begin{Def}
	应用评论重合率

	对于市场中的某一个评论列表为$CL_{app_k}$的应用$app_k$,假设其所有评论可以被分成$n$个组$CG_i (0 <i < n)$,则作者定义该应用的评论重合率$RD_{app_k}$如下。重合率越高,应用越有可能存在排名欺诈行为。
\end{Def}

\begin{equation}
	RD_{app_k} = 1 - \frac{\sum_{i=0}^n|CG_i|}{|CL_{app_k}|}
	\label{equ:cmt_simi1}
\end{equation}
\vspace{0.5mm}

为了找出内容高度重合的评论,研究者可以用NLP中的词袋模型(Bag-of-words Model)将每个评论转化成一串词语列表。
具体做法是,先对每条评论进行分词,形成词袋,再用一种合适的标准去衡量不同词袋之间的相似度,并将相似内容聚为一类。
分词方面,本研究使用了中文分词项目“结巴”中文分词~\footnote{\url{https://github.com/fxsjy/jieba}},为了更好地从词袋中提取语义信息,本文还从网上整理了一份停用词(Stop words)表,在分词之后筛去停用词,以减少不含语义的停用词对相似度计算造成的干扰。
词频太低的词语也可能会对聚类产生影响,为此,本研究会在聚类前从各个词袋中删除总词频太低的词语。
而在衡量词袋相似度方面,本文再次使用了\autoref{equ:usr_cre2}的Jaccard相关系数。

另外,本研究要筛查的是排名欺诈行为,其本质是通过提供大量虚假好评提高应用的平均评价,所以还要从数据集中除去一部分评论较少的应用,因为他们不太可能购买了排名欺诈服务。
本文分别从数据集中剔去了总评论少于50条和总评论少于100条的应用,使得数据组中分别剩下511和455个应用参与排名。

在预处理过程中,本研究剔除了在所有评论中出现次数小于等于2的词语,然后以0.8为相似度阈值对评论进行聚类。

结果可见\autoref{fig:cmt_simi},其中图例上标注的``50''和``100''分别表示剔除了总评论少于50条的应用的数据组和剔除了总评论少于50条的应用的数据组,两个图形中的三条虚线分别是两组数据中的3个四分位数线。
``50''数据组的三条四分位数线分别对应$x$轴上0.09,0.17和0.30的位置,说明数据组中有25\%的应用评论重合率小于9\%,50\%应用的评论重合率小于17\%,如果某应用的评论重合率大于30\%,那么该应用的评论重合率就排在数据集的前25\%了。
与之类似,``100''数据组的三条四分位数线分别对应$x$轴上0.10,0.18和0.32的位置,表明两组数据整体的评论重合率并不高。
此外,``100''组的数据分布比``50''组的数据分布稍微偏向$x$轴右侧,证明接收到评论较少的应用的评论重合率的确也偏低。

\begin{figure}[htbp]
	\centering
	\includegraphics[width=\textwidth]{./Figures/edwin-cmt-simi-dist.png}
    \caption{两种数据组的应用评论重合率结果}
    \label{fig:cmt_simi}
\end{figure}

\begin{table}[htbp]
	\renewcommand{\arraystretch}{1}
	\small
	\centering
	\caption{评论重合率结果}
	\vspace{1mm}
	\begin{tabular}{lcccc}
		\toprule
		包名 & 评论重合率(\%) & 排名(数据组``50'') & 排名(数据组``100'') \\
		\midrule
		com.arsenal.FunWeather & 47.11 & 58 & 57 \\
		\rowcolor{gray!15} com.jhwl.fjxa & 44.97 & 66 & 65 \\
		com.sscwap & 10.98 & 349 & 325 \\
		\bottomrule
	\end{tabular}
	\label{table:cmt-simi-res}
\end{table}

\autoref{table:cmt-simi-res}提供了三款仿冒应用的评论重合率和在两组数据中的排名情况。
在``50''数据组的511个应用中,\emph{com.arsenal.FunWeather}和\emph{com.jhwl.fjxa}分别排名58和66(前11\%和前13\%),而在``100''数据组的455个应用中,\emph{com.arsenal.FunWeather}和\emph{com.jhwl.fjxa}的排名是57和65(前13\%和前14\%),都算是比较靠前的位置。
而且,只看评论重合率,两款应用的数值都超过了44\%,相当于差不多每五条评论中就有两条十分类似的评论,这表明上述两款应用很有可能使用了排名欺诈。
至于\emph{com.sscwap}的排名则比较偏后,较低的重合率说明其好评比较多元化,使用排名欺诈服务的可能性比较低。
本研究会在后面一节对这些结果进行验证。

\subsection{人工复核}

最后,本研究对\emph{com.arsenal.FunWeather}、\emph{com.jhwl.fjxa}和\emph{com.sscwap}的评论了进行人工复核。
为了方便复核,作者对每个应用的评论都按内容进行了排序。

\begin{figure}[htbp]
	\centering
	\subfloat[\emph{com.jhwl.fjxa}部分评论\label{fig:cmt-sample-1}]{\includegraphics[width=\textwidth]{./Figures/edwin-cmt-sample-1.jpg}}\hfill

	\subfloat[\emph{com.arsenal.FunWeather}部分评论\label{fig:cmt-sample-2}]{\includegraphics[width=\textwidth]{./Figures/edwin-cmt-sample-2.jpg}}\hfill

	\subfloat[\emph{com.sscwap}部分评论\label{fig:cmt-sample-3}]{\includegraphics[width=\textwidth]{./Figures/edwin-cmt-sample-3.jpg}}\hfill
    \caption{应用评论取样}
    \label{fig:cmt-samples-1}
\end{figure}

\autoref{fig:cmt-samples-1}展示了本文对三款应用好评取样的结果。
可以明显看出,\emph{com.jhwl.fjxa}和\emph{com.arsenal.FunWeather}两款App都有明显的评论重复现象。
而图中对\emph{com.sscwap}的评论虽然看上去也比较近似,但这其实是由于本文按照评论内容对这些评论进行了一次排序。
如果结合日期数据观察,就能发现这些数据是由用户在比较分散的时间发出的。
所以说这些看上去稍微类似的评论,并不一定存在关联关系,作者不倾向于认为这款App购买了排名欺诈服务。
反观\autoref{fig:cmt-sample-1}中的评论,有多条长评论高度相似、甚至一模一样,这在真实案例中是不太可能会存在的情况;而\autoref{fig:cmt-sample-2}中多条近似、相同的评论是在十分接近的时间里被发表的,进一步加深了他们之间存在关联的可能性,所以本文十分有理由相信这两款应用的确存在排名欺诈的行为。

本研究还随机抽取了一些发布相同好评的不同用户进行追查,结果发现部分用户在研究收集到的数据集中的评论数只有一到两条,这很有可能是提供排名欺诈服务的商家规避检测的一种策略。

\section{本章小结}
本章从仿冒应用在市场中收到的评论入手,首先分析了用户对于这类应用的反响。
用户给出的反馈为一致好评,这十分出人意料。
进一步地,为验证仿冒应用与排名欺诈是否有关联,本文选取了部分应用的评论数据进行分析,并借鉴了前人研究提出了研究手段进行筛查。

一方面,从结果看,无论是基于评论内容相似度的排名欺诈排查方法还是基于用户可信度权重的排查方法,都证明了仿冒应用的确会利用排名欺诈服务提升自身的曝光率。
另一方面,用两种方法进行筛查也顺便比较了他们的有效性。
从性能层面看,两种方法有较大的区别。
研究发现,部分可疑好评的用户仅仅评论过一到两次,如果大部分排名欺诈用户都采用这种只发布少量好评的策略,基于用户可信度权重的排查方法就很可能会因为可参考的数据量太小而失效;
另外,基于用户可信度权重的排查方法会带来相当大的运算量,相比之下,基于评论内容相似度的代码只需几分钟就能运算完毕。
综合上述因素,本研究认为基于评论内容相似度的排查方法是更为可行的排查方式。

\clearPaperPage

\chapter{总结与展望}
\label{chp:future}

\section{总结}
% In this paper we first introduce the concept of fake apps, and study specifically towards these apps.
在本文中,笔者率先引入了``仿冒应用''这个概念,然后对这一方面进行了专门的研究,还搜集了大量的相关样本以辅助调查。
% To the best of our knowledge, we are the first to conduct a comprehensive empirical study on a large-scale fake apps.
据笔者所知,本课题是第一个针对仿冒应用进行大规模全面实证研究的课题。

% To better understand the ecosystem nature of this type of apps, we obtained more than 150,000 data entries from real-world markets, observed and measured the fake samples among this dataset from several dimensions including certificate information, app size, app name and package name, time factor and so on.
为了更好地了解这个类型的应用的生态环境,本文基于Python 3设计实现了仿冒应用筛选框架\mytool,利用基于BFS的算法从现实世界中的各个应用市场中获取了超过15万个应用样本,并且从多个不同维度,对这个数据库里面的仿冒样本进行了观测和考察。
这些维度包括了APK包中的安全证书信息、应用大小、应用名、包名和时间因素等等。

% Through our measurements we gain valuable experience on fake apps from several perspectives, findings like fake samples' naming tendency and fake developers' evasive strategies are inferred.
然后,本研究将收集到的数据分为了\emph{仿冒应用的基本特征}、\emph{影响仿冒应用数量的因素}和\emph{仿冒应用的发展轨迹}三个不同视角进行了测量,获得了不少宝贵的经验,并推断出了一些发现,比如仿冒应用的命名倾向和仿冒应用开发者对市场监管防御机制的规避策略。
% To support our findings, we further present a few study cases which provide us a more detailed look into fake apps to back our discoveries on fake app ecosystem.
为了佐证本文的发现,笔者进一步给出了从数据集中挑选的几个研究案例,呈现了如仿冒作者对不同热门应用的仿冒方式等内容。这几个案例进一步深化了本文对仿冒应用生态系统的发现。

之后,本文还收集了部分仿冒样本在商场上对应的评论和评级,进一步了解普通用户对这些仿冒应用持有的态度。
但在研究过程中,笔者发现仿冒应用受到的评价异常地高,于是在本研究中使用了两种研究方法---基于用户可信度和基于评论内容相似度的验证方法---对仿冒应用的评论进行了检测。
结果显示,刷好评的排名欺诈行为的确存在于应用市场中,在本研究搜集到的仿冒应用评论中就有刷好评的痕迹。

% We hope the lessons learned in this article are informative and helpful for mobile security practitioners in both academia and industry to improve the status quo.
笔者希望本文研究的结果能够为移动安全产业的从业人员(不论是在工业界还是学术界)提供足够的信息,以改善移动安全界的现状。

\section{展望}

在大规模分析的部分中,本文中用到了三个不同的角度分别探索仿冒应用的特征,但回顾探索过程,一些方法和步骤依然不够深入。
如果能从以下三个角度再向仿冒应用入手研究,或许能有更多有所裨益的发现:

\begin{itemize}
    \item 应用图标:
    本文在进行案例研究中发现,不少仿冒应用的图标和原版官方应用的图标其实十分相像。
    因此,图标也可以是一个用于发掘/鉴别仿冒应用的突破口,研究者也许可以从应用图标中挖掘到更多可用的信息与行为模式。
    碍于时间因素所限,本文研究中并未加入图像对比处理部分提取各APK包中的图标与官方应用的图标进行比对,但如果能研究出快速比对多个应用间图标、图像相似度的算法,定当对应用市场的安全监管筛选机制有所好处。

    \item 应用内代码/文本/链接/ip分析:
    代码分析可以有效地剖析应用的行为,而相似的文本资源、链接等信息也可以提供各个App之间可能存在的关联关系。
    遗憾的是,从当前技术水平出发,仔细地对一个App进行完整而全面的静态分析所需时长太长,而动态分析需要测试样例驱动,自动化的动态测试工具往往未能深入拓展一个应用的大部分核心功能。
    因此,开发出快速的分析算法对App进行更深入的探索,就能挖掘出有关仿冒应用生态的更多信息。

    \item 仿冒应用总量的变化原因:
    \fullref{chp:discoveries}中提及到了Janus收集到的仿冒应用数量并非一直保持上升趋势。
    近年来,能搜集到的仿冒应用数量有突然下跌、甚至渐渐式微的迹象。
    究竟是什么因素导致了这个原因?是移动黑灰产内部的变化,还是安全厂商日益紧密的封锁?
    这将会是一个十分有趣的课题。
\end{itemize}

而在评论分析的部分中,也有可以继续发掘的部分。
在现阶段,学界关于排名欺诈的研究一直在针对积极评价方面,但是对应用差评进行排名欺诈的相关研究却有待补充。
刷好评可以提高应用评价提升应用排名,如果反其道而行之,用差评对目标App进行攻击,其实也可以降低目标App的评价,对其排名进行打击。
另外,在实际上,用户给的差评中含有相当多的有用信息。
用户对应用的不满、功能上的建议、bug的反馈,都可以反映在差评上。
关于用户差评,还有很多的研究空间。

最后,\mytool 框架本身,无论是代码层面还是设计层面,也有值得改善的地方。
比如是否能利用自动化爬虫框架提高爬虫模块的鲁棒性(对抗应用商店的反爬虫技术、下载稳定性),工具本身的代码优化,还有工具整体的易用性、稳定性等。

总之,从整体上看,本文的工作还有很多可以深化的部分。
笔者希望本文能抛砖引玉,在让读者对移动应用黑色产业有更多认识的同时,激发读者对仿冒应用等方面的研究兴趣,并从上述几点出发,为后人带来更多深入而完善的相关研究。

\clearPaperPage

% 参考文献
\addcontentsline{toc}{chapter}{\small{参考文献}}
\bibliographystyle{GBT7714-2005}
\bibliography{bib/REFERENCE}
\clearPaperPage

% 致谢
\ifx\anonymous\undefined
	\fancypagestyle{plain}{%
		\fancyhead[LE,RO]{\small{\articleCategory}}
		\fancyhead[RE,LO]{\small{致  谢}}
	}
	\input{D2-ACKNOWLEDGEMENT.tex}
	\clearPaperPage
\fi

% 攻读学位期间发表过的论文
\input{D3-MYACHIEVEMENTS.tex}

\end{document}
