\phantomsection\addcontentsline{toc}{chapter}{\small{致  谢}}

\chapter*{致 \quad 谢}

在师大的时光转瞬而逝。不知不觉,研究生历程已快告一段落,下一个人生阶段也触手可及。
在就读研究生的三个春秋,鄙人有幸,承蒙了许多前辈的爱护与同侪的关照,借此机会聊表谢意。

首先、也是最想感谢的是我的导师徐立华教授。
徐老师学风严谨,逻辑清晰,总是会耐心细致地给予学生各方面的指引关怀,更从不吝向学生分享各种资源。
她给予我的,是学术上的引导,也是人生路上的指点。
此外,感谢贺樑教授在本文写作期间给我的指导与建议;感谢犇众信息的唐祝寿老师带我走近业界;感谢凯斯西储大学的肖旭生教授给予我学术访问的机会;也感谢各位曾给我授课或予我关怀的各位老师,你们的言传身教将会成为我宝贵的精神财富。

身边的各位同学也在各方面给了我很大的帮助。
感谢陈森、范玲玲两位博士带我完成了关于仿冒应用的研究。
同时,感谢谷林涛、卜文奇、袁宇杰、张雨、纪焘、汪庆顺、刘剑、叶莎莎、龚鑫、李雪莲等同学陪伴我一起度过了在华师大的研究生历程。
还有凯斯西储大学的刘简、杨劭、汪汉林、方鹏程、孙世宇、李昊、孟令萱和戴琪雨同学,半年美国访学之行虽短,但将是我人生其中一段最珍贵的回忆。

我还想感谢多年以来一直支持我的家人和朋友。
家人的养育之恩自不必说,每每遭遇人生路上的颠簸,都是家里的温暖让我重拾了前进的动力。
很感谢陈玄同学自高中以来的相伴。千金易得,知己难逢,希望我们友谊的小船能划得更远。

最后,感谢看到这里的你。

回首师大求学时光,当真如梦似幻。
七岁流转,纵有遗憾,但更多的是快乐与美好。
丽娃河上的点点碎金,樱桃河畔的轻风拂柳,都已深深地刻入我的记忆。
未来人生路长,惟愿铭记``求实创造,为人师表''的格言,勤勤恳恳,不负韶华,更不负诸位照耀过我的光。

\rightline{唐崇斌}
\rightline{二〇二〇年三月}
