\chapter{仿冒应用检测框架\mytool}
\label{chp:framework_prototype}

前两章分别从仿冒应用基本特征与仿冒应用开发者的行为特征入手,对仿冒应用开展实证研究,获取了仿冒应用命名、大小、投放偏好等特性。
基于这些特性,作者总结出了数条规则,设计了仿冒应用检测框架\mytool 。
本章将阐述\mytool 的设计与实现,并对后续的系统实验进行解析。

\section{框架设计与实现}

\subsection{整体设计}
\mytool 是一个轻量的基于规则的仿冒应用检测框架,具有可插拔式的规则配置模块,用户可根据实际向其中配置仿冒应用的检测规则,进行自定义筛选。
应用市场方可将其用于大规模的Android应用检测,自动筛选出其中的潜在仿冒应用,再结合人工审查,完成对仿冒应用的验证与归类,维护良好的应用市场环境。
此外,尽管\mytool 设计的初衷为检测应用市场中的仿冒应用,但基于其规则可插拔的特点,市场方的开发人员也可对其稍加改造后,配置其他规则,以进行针对重打包应用、恶意应用等移动灰黑产的检测。

% todo: 框架流程配图

展示了\mytool 的整体工作流程,框架主要分为\componentA 、\componentB 、\componentC 和人工审查四个部分。

\mytool 以Android应用集为输入,先利用\componentA 对应用集进行初筛,过滤出、,

\subsection{\componentA }
\subsection{\componentB }
\subsection{\componentC }
\subsection{人工审查与特征提取}
\section{系统实验}
\subsection{数据收集}
\subsection{检测结果}
\section{本章小结}
