% 前置\cleardoublepage\phantomsection 以确保目录超链接到正确页码
\cleardoublepage\phantomsection\addcontentsline{toc}{chapter}{摘要}

\chapter*{\zihao{2}\heiti{摘~~~~要}}
\vspace{-5mm}

\setlength{\baselineskip}{25pt} % 25磅行距

Android移动操作系统由于其开源、用户广泛等特点,拥有庞大的应用生态环境,其中包含各种良性应用,以及恶意应用、重打包应用和仿冒应用。
得益于广泛的前人研究,业界对各类恶意、重打包应用具备充分理解,衍生出了对应检测机制与防护措施;
然而,近年来Android安全相关研究多聚焦于恶意、重打包应用,学术界对仿冒应用进行的研究较为匮乏,Android移动应用安全仍有隐患。

目前,针对仿冒应用的研究面临如下三点问题与挑战:
(1)业界尚缺乏一个公开可用的大规模仿冒应用数据集供研究者使用。其难点在于:收集数据时需要同时考虑数据规模、来源多样性与代表性;
(2)业界对仿冒应用的概念较为模糊,仿冒应用的基本特征尚未明晰,进而阻碍后续的检测工作;
(3)仿冒应用作者的行为特征尚不明确,无法为业界预防或拦截仿冒应用提供有效指导。

为获得关于仿冒应用的原始资料,本文采取实证研究的方式,从以下四个方面对仿冒应用进行了深入的分析和研究:

\textbf{通过收集大规模仿冒应用样本集,改善仿冒应用数据稀缺问题。}
本研究从应用宝、豌豆荚等多个应用来源入手爬取数据,最终筛选出5万个余个仿冒应用,可为后续工作如应用特征提取、应用行为分析等提供数据支持。

\textbf{针对Android仿冒应用特征进行多方面分析研究。}
利用上述数据集,本文对仿冒应用进行了与原版应用的相似度比对、功能比对等多方面解析,提供了仿冒应用的基本特征解读结果,并提供案例分析。
分析结果可为普通用户与应用市场等群体抵御仿冒应用威胁提供指导。

\textbf{利用证书信息与应用发布时间提供应用开发者行为画像。}
结合仿冒样本发布时间和从仿冒样本中抽取出的应用证书信息,本文从活跃期、投放特征等角度对仿冒应用开发者开展了一系列行为画像,该画像反映的行为特征可为应用市场提供监管思路。

\textbf{提出了可用的仿冒应用排查框架。}
实证研究表明,国内应用市场尚未有应对仿冒应用的完善措施。
针对该现状,本文推出了检测框架\mytool ,帮助应用市场在大规模应用中实现对已知正版应用及其对应仿冒样本的快速鉴别。

综上,本研究收集了较大规模数据集,可为后续工作提供数据支持和探索方向。
同时,依托收集到的大量数据,本研究对仿冒应用进行了全面分析,对普通用户和应用市场均提出了对应的实用建议,并提出了可行的检测方案,有助于提升各方对仿冒应用的认识与重视。

\sihao{\heiti{ 关键词:}} \xiaosi{Android应用程序, 仿冒应用, 实证研究, 数据分析}
