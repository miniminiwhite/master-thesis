\chapter{总结与展望}
\label{chp:future}
\section{总结}


%本文提出了一个基于GUI的自动化性能测试方法,并通过Java实现了其功能,开发了自动化性能测试框架PerDroid。
本文提出并实现了一个可用于生成Android应用程序动态函数调用图的技术方案——RunDroid。
RunDroid生产的函数调用图,可以准确的反映应用程序的执行过程,取得了有意义的研究成果。
本文的主要工作如下:


1) :本文在传统的函数调用概念的基础上,提出了函数触发关系,用于描述两个方法之间的因果关系;
并结合方法触发关系、方法对象等概念提出了拓展函数调用图的概念。

2) RunDroid的设计与实现:
RunDroid利用源程序代码插桩和运行时方法拦截相结合的方式,获取应用方法执行信息,构建函数调用图;并在此基础上,利用方法和对象的关系补全到调用图中的方法间触发关系,展现运行过程中的Android特性行为。


3)静动态工具的实验结果对比:
本文将RunDroid产生的动态函数调用图和FlowDroid产生的静态函数调用图进行对比。
相比FlowDroid,RunDroid产生的函数调用图能够体现应用程序的执行过程,表现函数间的调用关系和触发关系,准确地还原Android组件的生命周期。
%就Activity 生命周期及事件回调、多线程触发关系等角度和与传统静态分析工具FlowDroid产生的静态调用图做了对比分析,并分析两种技术的优劣。

4)开源应用的统计实验:利用RunDroid构建开源Android应用的动态函数图,统计数据佐证了事件回调、Handler等函数间触发关系在Android应用中的普遍性。

%反映方法间触发关系在Android系统上的常见性。


5)RunDroid在错误定位领域的应用:
相关实验结果实现,从RunDroid提供的函数关系信息可以反映出更多程序依赖信息,
相比之前技术方案,方法间的因果关系模型更健全,实验的可靠性有所提升。
%这使得我们的工具具备一定的学术研究价值。


\section{展望}

虽然通过RunDroid还原得到的Android应用程序动态函数调用图,反映程序的运行时状态,但在实验过程中我们发现以下问题:

1)RunDroid在捕获应用用户层方法时,采用的方案是源代码插桩方案。
调用图构建的前置条件需要提供Android 应用的源代码,因此,RunDroid的运行对源代码高度依赖。

2)在实验阶段,我们发现,当应用程序长时间运行时,应用程序会产生较多的日志。
通常的,移动设备上的存储是有限的。
因此,对于一些调用关系较为复杂的应用,RunDroid的日志方案比较容易遇到日志存储的瓶颈。

3)RunDroid中的运行时拦截器是基于Xposed框架实现的。Xposed框架并不是适用于所有的Android手机,在一定程度给RunDroid的实验环境提出了额外的要求。

整体上,本文提出的RunDroid较为准确地还原出Android应用程序在运行过程的函数调用图。
针对RunDroid实验中发现的不足,RunDroid的后续工作可以从以下几个方面进行改进:
利用字节码修改技术代替源代码修改方案以减少RunDroid运行过程中对源代码的依赖;
引入基于JVMTI的调试环境,借助调试技术实现系统方法执行的拦截,摆脱对Xposed环境的依赖;
通过静态分析技术确定运行过程的确定性路径,缩减待插桩的用户方法数量,进而减少运行时日志的产出量。