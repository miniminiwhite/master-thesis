\chapter{总结与展望}
\label{chp:future}

\section{总结}
本文率先引入了``仿冒应用''这一概念。
在在搜集了大量数据的前提下,对仿冒应用从两个不同角度先后进行了两次实证研究,获取了仿冒应用的基本特性和仿冒应用开发者的行为特征,并推出了仿冒应用检测框架\mytool ,有一定现实意义。
具体地,本文主要工作总结如下:

1)引入了``仿冒应用''的概念,对仿冒应用的危害进行了简要分析,并借助Janus对收集了来自现实世界中29个应用来源的近14万个应用样本,获取了样本的证书信息、应用大小、应用名等信息,组成数据集。

2)利用上述数据集,分别从仿冒应用与原版应用相似度、影响应用被仿冒的严重程度的因素及仿冒应用行为入手开展实证研究,结合实际案例,对仿冒应用的基本特征进行分析,推断出``仿冒应用作者更倾向重新制作仿冒应用,而非通过重打包方式制作仿冒应用''的结论。

3)利用数据集中的另一部分数据,完成面向仿冒应用开发者行为特征的实证研究,透过开发者可利用Android调试证书上传应用、仿冒应用开发者证书可长时间活跃等现象,推断出现有应用市场在监管方面的不足。

4)在完成上述两次实证研究后,有针对性地分别向普通用户、开发者与应用市场方三个不同群体提出了切实可行的实用建议。

5)设计并实现了仿冒应用检测框架\mytool ,利用系统实验说明了\mytool 的可用性。


\section{展望}

本研究致力于展现一个面向仿冒应用的清晰视角,并通过设计\mytool 协助市场方提升仿冒应用对仿冒应用的拦截能力。
然而,在研究中仍有几点问题,可作为未来工作的方向。
其一为对仿冒应用基本特征的实证研究中未涉及与代码、应用行为相关的具体研究,若能分析仿冒应用的行为并总结出一类特征,将十分有助于后续检测。
其二为\mytool 中\componentB 的代码信息获取部分,现时方案还无法追踪用户的Java反射调用相关代码;\mytool 的图标匹配部分也有待加强以提高灵敏度。
其三为\mytool 仍对人工有较强依赖,包括对输出的结果进行确认判断和规则的分析提取。尽管人工审核在安全领域难以避免,但在未来工作中或许可为\mytool 添加一个自动分析模块,减少在特征分析上的所需人力。
